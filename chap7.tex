\chapter{Time dependent PT}
\label{chapt7}

\section{Introduction}

Up to this point all the systems we have solved are time independent, i.e the Hamiltonian has no time dependence $\Hat{H} \neq \Hat{H}(t)$, such that the potential energy operator, $\Hat{V}$, is only a function of position and not time: $\Hat{V} = \Hat{V}(\underline{r}).$ This is known as quantum statics. This means that the Time Dependent Schrodinger Equation (TDSE):

\begin{equation}
    i\hbar \frac{\partial}{\partial t} \ket{\Psi(\underline{r}, t)} = \Hat{H}(\underline{r}) \ket{\Psi(\underline{r}, t)}
    \label{TDSE},
\end{equation}

\noindent can be solved using separation of variables, such that we can write $\Psi(\underline{r}, t) = \psi(\underline{r}) \phi(t)$, where $\psi(\underline{r})$ satisfies the Time Independent Schrodinger Equation (TISE):

\begin{equation}
    \Hat{H}(\underline{r}) \ket{\Psi(\underline{r})} = E \ket{\Psi(\underline{r})}
    \label{TISE}
\end{equation}

\noindent and $\phi(t) = e^{\frac{-iEt}{\hbar}}$, with $E$ the energy eigenvalue corresponding to the eigenstate $\ket{\Psi(\underline{r})}$. However in this section we consider some basic quantum dynamics, i.e $\Hat{V} = \Hat{V}(\underline{r}, t)$.

\section{Derivation}

\subsection{Non-Stationary States}

Recall that for a quantum static system, if the system occupies an eigenstate of the time independent Hamiltonian, $\Hat{H}_0$, then it is said to be in a stationary state of the system. This just means that no matter what time you measure the system it is always in the same eigenstate with the same eigenvalue, i.e initially it's in the state $\ket{\psi_m}$ and then at some later time $t$ in state $\ket{\psi_m} e^\frac{-iE_m t}{\hbar}$ which are equivalent once a measurement is applied.

\noindent However now consider what happens when the same quantum static system is not in a stationary state of the system, $\ket{\Psi(\underline{r}, t)}$. Suppose at $t = 0$s, the state of the system may be written as a linear combination of the complete set of eigenstates that the system can occupy:

\begin{equation}
    \ket{\Psi(\underline{r}, t = 0)} = \sum_n = c_n \ket{\psi_n},
    \label{LinearCombinationStatesNonStationaryNoTimeDependence}
\end{equation}

\noindent such that:

\begin{equation}
    \ket{\Psi(\underline{r}, t)} = \sum_n = c_n \ket{\psi_n} e^\frac{-i E_n t}{\hbar}.
    \label{LinearCombinationStatesNonStationaryTimeDependence}
\end{equation}

\noindent This linear combination of stationary states means that we have a system described by a Hamiltonian which is inherently time independent that can undergo transitions (or quantum jumps) between different states assuming the system doesn't start in a stationary state.

\noindent This should become more clear after the following 2-level system example. Consider a system with Hamiltonian, $\hat{H}_0$, which is initially in the state:

\begin{equation}
    \ket{\Psi(\underline{r}, t = 0)} = c_m \ket{\psi_m} + c_n \ket{\psi_n},
    \label{TwoLevelSystemInitialState}
\end{equation}

\noindent where $\ket{\psi_m}$ and $\ket{\psi_n}$ are the orthonormal eigenstates corresponding to the two levels of this system, with eigenvalues $E_m$ and $E_n$ respectively. This means that in the absence of any perturbation we have:

\begin{equation}
    \ket{\Psi(\underline{r}, t)} = c_m \ket{\psi_m} e^\frac{-i E_m t}{\hbar} + c_n \ket{\psi_n} e^\frac{-i E_n t}{\hbar},
    \label{TwoLevelSystemGeneralState}
\end{equation}

\noindent where we note that normalisation requires $|c_m|^2 + |c_n|^2 = 1$. Consider the expectation value of this system with respect to some arbitrary time independent operator $\Hat{A}$:

\begin{equation}
    \bra{\psi}\Hat{A}\ket{\psi} = \langle a \rangle _t = |c_m|^2 A_{mm} + |c_n|^2 A_{nn} + 2 \Re e[c_m^* c_n A_{mn} e^{i\omega t}],
    \label{TwoLevelSystemExpectationValue}
\end{equation}

\noindent where $\ket{\psi}$ is given by Eq~\ref{TwoLevelSystemGeneralState}, $\omega = \frac{E_m - E_n}{\hbar}$, $A_{mn} = \bra{\psi_m}\Hat{A}\ket{\psi_n}$ and we use the fact that $\Hat{A}$ is hermitian. Note the subscript $t$ on $\langle a \rangle$ denotes that the expectation value varies with time. From Eq~\ref{TwoLevelSystemExpectationValue} we can see that $\langle a \rangle _t$ oscillates with a period $T = \frac{2\pi}{\omega}$ between:

\begin{equation}
    |c_m|^2 A_{mm} + |c_n|^2 A_{nn} \pm 2 c_m^* c_n A_{mn},
\end{equation}

\noindent since $\Re e[e^{i \omega t}] = \cos(\omega t)$. This oscillation represents the transitions or quantum jumps between eigenstates.

\subsection{Time Dependent Perturbation Theory}

In general to allow the quantum jumps between energy eigenstates we saw in the previous section we can allow $\Hat{H} = \Hat{H}(\underline{r}, t)$. This is normally done by introducing a time dependent potential, $\Hat{V}(t)$, which we can treat as a perturbation if $\Hat{V}(t) \ll \Hat{H}_0$. This is known as time dependent perturbation theory.

\subsection{Emission and Absorption of EM radiation}

\section{Key Results}

\section{Exam Question?}


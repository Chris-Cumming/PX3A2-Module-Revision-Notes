\chapter{SOC and the Zeeman effect}
\label{chapt4}

\section{Introduction}

This entire section is effectively a massive example of how useful degenerate PT is. Here we apply it to a hydrogenic system to quantify the magnitude of energy splittings when we apply an external magnetic field, allow spin-orbit coupling (SOC) or have a combination of both. This is done both using the brute force method, described in Section~\ref{chapt2}, and by figuring out what the correct eigenstates for PT are using symmetry, which I'll explain briefly in a minute.

\noindent Unfortunately the notation isn't great in places here, because in some cases V means the scalar potential of the nucleus and in others it means the potential energy of the electron in the scalar potential of the nucleus. I've tried to make it clear here using $V_c$ meaning the potential energy. There's also some funny business going on with operator symbols.

\section{Derivation}

The Zeeman effect involves an external magnetic field coupling to the magnetic moments of an electron and SOC involves the coupling between these magnetic moments. In general the magnetic moment of a particle due to orbital motion is:

\begin{equation}
    \underline{\mu}_L = g_L \frac{q}{2m} \underline{L},
    \label{OrbitalMagneticMoment}
\end{equation}

\noindent where $g_L$ is the orbital gyromagnetic ration, $q$ is the charge of the particle, $m$ the mass and $\underline{L}$ the orbital angular momentum vector. However all particles also have intrinisc angular momentum (spin), which also gives rise to a magnetic moment:

\begin{equation}
    \underline{\mu}_S = g_S \frac{q}{2m} \underline{S},
    \label{SpinMagneticMoment}
\end{equation}

\noindent where $g_S$ is the spin gyromagnetic ratio and $\underline{S}$ the spin angular momentum vector.

\noindent For an electron $g_L = 1$, $g_S = 2$ and $q = -e$. Note that $g_S$ doesn't exactly equal 2 there are some corrections to this value due to QED, which is also a perturbation theory but in the context of quantum field theory. 

\noindent Now if we are going to apply degenerate PT to a hydrogenic system that experiences the Zeeman effect and SOC, we need to know the forms of the perturbation, $\Hat{V}$, we apply. For the Zeeman effect, the operator is found from the interaction energy between $\underline\mu$ and $\underline{B}$:

\begin{equation}
    \Hat{V}_{mag} = - \underline{\Hat{\mu}} \cdot \underline{\Hat{B}} = \frac{e}{2m}(\Hat{L}_z + 2\Hat{S}_z) \cdot B,
    \label{ZeemanEffectPerturbation}
\end{equation}

\noindent where we assume that $\underline{B} = (0, 0, B)$, i.e we align the magnetic field along the $\Hat{z}$ direction, and $\Hat{L}_z$ and $\Hat{S}_z$ are the normal operators for returning the z component of the orbital and spin angular momentum.

\noindent Unfortunately the operator for the SOC perturbation is a tad more complicated. In the rest frame of the electron, the nucleus is orbiting it. This means the electron "sees" a magnetic field due do the motion of the charged nucleus. It is given by:

\begin{equation}
    \underline{B} = \frac{-\underline{v} \times \underline{E}(r)}{c^2},
    \label{MagneticFieldSOC}
\end{equation}

\noindent where $\underline{v}$ is the relative velocity of the nucleus, $\underline{E}(r)$ is the electric field due to the nucleus and $c$ is the speed of light. Since the electric field of the nucleus is spherically symmetric we may write it as:

\begin{equation}
    \underline{E}(r) = -\underline{\nabla} V = -\frac{\partial V}{\partial r} \Hat{\underline{r}},
    \label{SphericalElectricField}
\end{equation}

\noindent where $V$ is the scalar electric potential of the nucleus. The operator for SOC takes the form:

\begin{equation}
    \Hat{V}_{SOC} = - \frac{1}{2} \underline{\Hat{\mu}}_s \cdot \underline{\Hat{B}},
    \label{SOCPerturbation}
\end{equation}

\noindent where the factor of $\frac{1}{2}$ arises from the Thomas precession which is a relativistic effect. So substituting in Eq~\ref{SpinMagneticMoment} and Eq~\ref{MagneticFieldSOC} into the above equation we find:

\begin{equation}
    \Hat{V}_{SOC} = \frac{1}{2} \frac{e}{m} \frac{\underline{\Hat{S}} \cdot (\underline{v} \times \underline{r})}{c^2} \frac{1}{r} \frac{\partial V}{\partial r},
\end{equation}

\noindent where we have used that the radial unit vector, $\underline{\Hat{r}} = \frac{\underline{r}}{r}$. Then using the relations $\underline{p} = m \underline{v}$, $\underline{\Hat{L}} = \underline{\Hat{r}} \times \underline{p} = - \underline{p} \times \underline{\Hat{r}}$ and $\underline{\Hat{S}} \cdot \underline{\Hat{L}} = \underline{\Hat{L}} \cdot \underline{\Hat{S}}$, we find:

\begin{equation}
    \frac{-e}{2m^2c^2} \frac{1}{r} \frac{\partial V}{\partial r} \underline{\Hat{L}} \cdot \underline{\Hat{S}}.
    \label{SOCPerturbationForm1}
\end{equation}

\noindent There are a number of other forms that this operator can take, the most useful of which is found using:

\begin{equation}
    \Hat{J}^2 = (\Hat{L} + \Hat{S})^2 = \Hat{L}^2 + \Hat{S}^2 + 2 \Hat{L} \cdot \Hat{S},
    \label{TotalAngularMomentumExpansion}
\end{equation}

\noindent in combination with Eq~\ref{SOCPerturbationForm1}:

\begin{equation}
    \Hat{V}_{SOC} = f(r)[j(j + 1) - l(l + 1) - s(s + 1)],
    \label{SOCPerturbationForm2}
\end{equation}

\noindent SHOULD PROBABLY CHANGE THIS AVOID CONFUSION AS IS TECHNICALLY WRONG. where it should be noted that the square brackets here don't mean commutator they are just brackets, and $f(r)$ is given by:

\begin{equation}
    f(r) = \frac{\hbar^2}{4m^2c^2} \frac{1}{r} \frac{\partial V_c}{\partial r},
    \label{f(r)Form}
\end{equation}

\noindent where $V_c(r)$ is the potential energy of the electron in the nucleus' electric field, $\frac{-Ze^2}{4\pi\epsilon_0 r}$.

\subsection{Correct eigenstates using symmetry}

As I said at the end of Section~\ref{chapt2}, there is a more elegant way of doing degenerate PT. Instead of forming a matrix and finding its eigenvalues and corresponding eigenstates, we can figure out the correct eigenstates to use for degenerate PT using symmetry arguments.

\noindent The correct eigenstates for degenerate PT are ones that commute with the unperturbed hamiltonian, $\Hat{H}_0$, the perturbation we apply, $\Hat{V}$, and an arbitrary operator, $\Hat{A}$, that we are free to choose. Once these eigenstates have been identified you can just use Eq~\ref{FirstOrderEnergyCorrectionsDegenerate} to calculate the first order energy corrections for a degenerate system. Remember this expression is just the equivalent expression for Eq~\ref{FirstOrderEnergyCorrection} that accounts for degeneracy.

\noindent The reason this works is because any operators that commute with one another must share a set of common eigenstates. So if $[\Hat{H}_0, \Hat{A}] = [\Hat{V}, \Hat{A}] = 0$, then they all share a common set of eigenstates. Then so long as all the eigenvalues of $\Hat{A}$ are unique, this automatically leads to the diagonalisation condition, Eq~\ref{MatrixCondition}, being satisfied. This in turn means that those eigenstates are the correct unperturbed states for degenerate PT.

\noindent I wouldn't worry too much about trying to understand why this works, just know that the correct eigenstates for degenerate PT are ones that are eigenstates of $\Hat{H}_0$, $\Hat{V}$ and another operator, $\Hat{A}$.

\subsection{Spin-Orbit Coupling}

We'll start off by considering the case of just SOC, so no external magnetic field has been applied. I'll start off with the symmetry argument method. Since we are considering a hydrogenic system, i.e a single electron orbiting a nucleus, the eigenstates of the unperturbed system only depend on the principal quantum number, $n$. We will use Eq~\ref{SOCPerturbationForm2} as our perturbation, which requires eigenstates that depend on $l, s$ and $j$. Now we choose our other operator, $\Hat{A}$, to be $\Hat{J}_z$. This is just the total angular momentum equivalent of $\Hat{L}_z$ or $\Hat{S}_z$, i.e $\Hat{J}_z = \Hat{L}_z + \Hat{S}_z$. This means the correct eigenstates for degenerate PT look like $\ket{n, l, s, j, m_j}$. So now we can figure out the first order energy corrections using these eigenstates and Eq~\ref{FirstOrderEnergyCorrectionsDegenerate}:

\begin{equation}
    E^1_{SOC} = \braket{n, l, s, j, m_j | \Hat{V}_{SOC} | n, l, s, j, m_j},
    \label{FirstOrderEnergySOC}
\end{equation}

\noindent then inserting Eq~\ref{SOCPerturbationForm2} and Eq~\ref{f(r)Form}, we find:

\begin{equation}
    E^1_{SOC} = \frac{\hbar^2}{4m^2_ec^2}[j(j + 1) -l(l + 1) - s(s + 1)] \braket{\frac{1}{r}\frac{\partial V_c}{\partial r}},
    \label{FirstOrderEnergySOCForm1}
\end{equation}

\noindent where again the square brackets are just brackets and don't mean commutator, and the expectation value of $\frac{1}{r} \frac{\partial V_c}{\partial r}$ depends on which $(n,l)$ orbital the electron occupies. You won't be asked to calculate what this expectation value is since its quite hellish to derive, but simplifying the above expression we find:

\begin{equation}
    E^1_{SOC} = \frac{|E_n| \alpha^2}{n}(\frac{1}{l + \frac{1}{2}} - \frac{1}{j + \frac{1}{2}})
    \label{FirstOrderEnergySOCForm2}
\end{equation}

\noindent where $\alpha$ is the fine structure constant which equals $\frac{1}{137}$ and $E_n$ are the energy levels of a hydrogenic system, Eq~\ref{HydrogenicAtomEnergyLevels}. Normally the factor in front is just written as $\lambda$.

\noindent From this equation we see that if we are in the $s$ orbital of an energy level, i.e $l = 0$, then $E^1_{SOC} = 0$ since $j = s = \frac{1}{2}$, meaning the energy levels remain the same. However if $l \neq 0$ then each orbital is split into 2 levels, given by $j = l \pm \frac{1}{2}$, with each level being $2j + 1$ degenerate.

\noindent For the $p$ orbital, $l = 1$, then $j$ may either be $\frac{3}{2}$ or $\frac{1}{2}$, with the corresponding corrections being $E^1_{SOC} = \frac{\lambda}{6}$ and $E^1_{SOC} = \frac{-\lambda}{2}$. INSERT FIGURE.

\noindent This problem can also be solved using the brute force method, here the degenerate eigenstates are described by $\ket{m_l, m_s}$. This time we use a different form of the perturbation for SOC, we take Eq~\ref{SOCPerturbationForm1} and expand out $\underline{\Hat{L}} \cdot \underline{\Hat{S}}$ in terms of the raising and lowering operators, so that:

\begin{equation}
    \Hat{V}_{SOC} = \delta \underline{\Hat{L}} \cdot \underline{\Hat{S}} = \delta [\Hat{L}_z \Hat{S}_z + \frac{1}{2}(\Hat{L}_+ \Hat{S}_- + \Hat{L}_- \Hat{S}_+)],
    \label{PerturbationSOCFormMatrix}
\end{equation}

\noindent where again the square brackets don't mean commutator. As we described in Section~\ref{chapt2}, we form the perturbation matrix using our degenerate eigenstates $\ket{m_l, m_s}$ and the perturbation given by Eq~\ref{PerturbationSOCFormMatrix}. At this point its very important to remember that the degenerate eigenstates are orthonormal to each other, Eq~\ref{DegenerateEigenfunctions}, as well as the effect that the raising and lowering operators have on eigenstates. Note that the raising and lowering operators will be given to you in the exam.

\noindent Then calculating the eigenvalues of this matrix, which correspond to the first order energy corrections, we find that we get $E^1_{SOC} = \frac{\delta}{2}$ four times and $E^1_{SOC} = -\delta$ two times. This result matches up with that found using the correct unperturbed eigenstates, as we would expect.

\subsection{Strong Field Zeeman Effect}

If we apply an external magnetic field to a system, we will get the Zeeman effect. If this applied field is larger relative to the SOC field, i.e $\Hat{V}_{mag} \gg \Hat{V}_{SOC}$, then we are in the regime of the Strong field Zeeman effect. This means we can neglect any effects due to SOC and consider our unperturbed system as just the hydrogenic system again. In this case we use the neat method to find the first order energy corrections. The correct unperturbed eigenstates are given by $\ket{n, l, m_l, m_s}$, with our perturbation described by Eq~\ref{ZeemanEffectPerturbation}, so  the first order energy corrections are found using Eq~\ref{FirstOrderEnergyCorrectionsDegenerate} to be:

\begin{equation}
    E^1_{mag} = \mu_B B_z (m_l + 2m_s),
    \label{ZeemanEffectEnergyCorrections}
\end{equation}

\noindent where $\mu_B$ is the Bohr Magneton, given by $\frac{e\hbar}{2m_e}$. Note this is also known as the Paschen-Back effect. 

INSERT FIGURE HERE

\subsection{Weak Field Zeeman Effect}

Finally we consider what happens when we apply a small external magnetic field such that $\Hat{V}_{SOC} \gg \Hat{V}_{mag}$. This means that our unperturbed system includes SOC, i.e the unperturbed Hamiltonian is $\Hat{H}_0 + \Hat{V}_{SOC}$, with the external magnetic field acting as the perturbation. This problem again can be solved either with the brute force method or using the correct unperturbed eigenstates directly.

\noindent The correct unperturbed eigenstates are, as before, $\ket{n, l, s, j, m_j}$ with the perturbation given by Eq~\ref{ZeemanEffectPerturbation}, so using Eq~\ref{FirstOrderEnergyCorrectionsDegenerate} and that $\Hat{L}_z + 2\Hat{S}_z = \Hat{L}_z + \Hat{S}_z + \Hat{S}_z = \Hat{J}_z + \Hat{S}_z$, we find that the first order energy corrections take the form:

\begin{equation}
    E^1_{mag} = m_j g_L \mu_B B,
    \label{WeakFieldEnergyCorrections}
\end{equation}

\noindent where $g_L$ is the Lande $g$ factor:

\begin{equation}
    g_L = 1 + \frac{j (j + 1) + s (s + 1) - l (l + 1)}{2j(j + 1)}
    \label{LandGFactor}.
\end{equation}

\noindent This means that each energy level is split into $2j + 1$ energy levels separated by $g_L \mu_B B$.

\noindent Alternatively we can solve this using the brute force method. Here the degenerate eigenstates are once again $\ket{m_l, m_s}$, and we use the unmodified version of Eq~\ref{ZeemanEffectPerturbation} as our perturbation. Here we find all the corresponding perturbation matrix elements and add them to the original matrix we had when considering just SOC. Then as per usual we find the eigenvalues of this matrix to find the first order energy corrections, which are found to be: $\frac{\delta}{2} + 2\mu_B B_z$, $\frac{\delta}{2} - 2\mu_B B_z$, $\frac{\delta}{2} + \frac{2}{3}\mu_B B_z$, $-\delta + \frac{1}{3}\mu_B B_z$, $\frac{\delta}{2} - \frac{2}{3}\mu_B B_z$ and $-\delta - \frac{1}{3}\mu_B B_z$ for an $l = 1, s = \frac{1}{2}$ system.


\noindent Can be very useful to split up matrices into smaller matrices and then find the determinants of these, as a lot of terms will be 0, so can consider 1x1, 2x2, etc determinants.

\section{Key Results}





\section{Exam Question?}



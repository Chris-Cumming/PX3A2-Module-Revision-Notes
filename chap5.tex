\chapter{Identical Particles}
\label{chapt5}


\section{Introduction}

It is impossible to follow and identify one particular particle when dealing with a quantum system of identical particles. This is because any measurement we wish to make of the system will disturb it and all the particles are indistinguishable from one another.

\noindent The wave function for a system which contains N indistinguishable particles is $\psi(\underline{r}_1, \underline{r}_2, ..., \underline{r}_k, ..., \underline{r}_N, t)$, where $\underline{r}_k$ is the position vector for the $k^{th}$ particle. This has the same normalisation condition as for a single particle state:

\begin{equation}
    \braket{\psi | \psi} = 1 = \int | \psi(\underline{r}_1, \underline{r}_2, ..., \underline{r}_k, ..., \underline{r}_N, t) |^2 d^3\underline{r}_1 d^3\underline{r}_2 ... d^3\underline{r}_k ... d^3\underline{r}_N
    \label{ManyParticleNormalisation}
\end{equation}

\noindent This says that at some time t, there is a particle within $d^3\underline{r}_1$ of $\underline{r}_1$ and another particle within $d^3\underline{r}_2$ of $\underline{r}_2$, etc and not that there is a specific particle, since we can't distinguish between them.

\noindent Operators acting on multi-particle eigenstates must be symmetric under the permutation of particle labels, due to this indistinguishability. An example of this is the Hamiltonian for the two electrons orbiting a Helium nucleus:

\begin{equation}
    \Hat{H}(1, 2) = \sum_{i=1}^2 (\frac{-\hbar^2}{2m}\nabla^2_i - \frac{2e^2}{4 \pi \epsilon_0 r_i}) + \frac{e^2}{4 \pi \epsilon_0 r_{12}}
    \label{HeliumAtomHamiltonian}
\end{equation}

\noindent The indistinguishability of the electrons means that:

\begin{equation}
    \Hat{H}(1, 2) = \Hat{H}(2, 1)
    \label{ExchangeDegeneracy}
\end{equation}

\noindent this means that the eigenvalues are degenerate, this is what is known as exchange degeneracy.

\section{Particle Exchange Operator}



\section{Fermions and Bosons}


\section{Key Results}



\section{Exam Question?}



%----------------------------------------------------------------------------------------
% DOCUMENT CONFIGURATIONS
%----------------------------------------------------------------------------------------

\documentclass[
11pt, % The default document font size, options: 10pt, 11pt, 12pt
oneside, % Two side (alternating margins) for binding by default, uncomment to switch to one side
english, % other languages available
% singlespacing, % Single line spacing, alternatives: onehalfspacing or doublespacing
onehalfspacing, % Single line spacing, alternatives: onehalfspacing or doublespacing
%draft, % Uncomment to enable draft mode (no pictures, no links, overfull hboxes indicated)
%nolistspacing, % If the document is onehalfspacing or doublespacing, uncomment this to set spacing in lists to single
%liststotoc, % Uncomment to add the list of figures/tables/etc to the table of contents
%toctotoc, % Uncomment to add the main table of contents to the table of contents
]{McMasterThesis} % The class file specifying the document structure


%----------------------------------------------------------------------------------------
% Import packages here
%----------------------------------------------------------------------------------------
\usepackage[utf8]{inputenc} % Required for inputting international characters
\usepackage[T1]{fontenc} % Output font encoding for international characters

\usepackage{lmodern} % could change font type by calling a different package
\usepackage{lastpage} % count pages
\usepackage{siunitx} % for scientific units (micro-liter, etc)
\usepackage{xlop} %For arithmetic operations in a table.

\setcounter{tocdepth}{2} % so that only section and sub sections appear in Table of Contents. Remove or set depth to 3 to include sub-sub-sections


\usepackage{braket}
\usepackage{amsmath}



\usepackage{rotating} % rotating table and figure
\usepackage[T1]{fontenc}
\usepackage[utf8]{inputenc}
 
%% Text box -----------------------
\usepackage{mdframed} % text box
% all 4 borders
\newmdenv{allfour}

% just top and bottom
\newmdenv[leftline=false,rightline=false]{topbot}

% just left and bottom
\newmdenv[topline=false,rightline=false]{leftbot}


%% Stylist Text Boxe
\usepackage[many]{tcolorbox}
% \definecolor{main}{HTML}{5989cf}    % setting main color to be used
% \definecolor{sub}{HTML}{cde4ff}     % setting sub color to be used

\tcbset{
    sharp corners,
    % colback = white,
    before skip = 0.2cm,    % add extra space before the box
    after skip = 0.5cm      % add extra space after the box
}                           % setting global options for tcolorbox
\newtcolorbox{boxK}{
    center,
    width=5.2in,
    % height=0.9in,
    % sharpish corners, % better drop shadow
    boxrule = 0pt,
    toprule = 0pt, % top rule weight
    % enhanced,
    % fuzzy shadow = {0pt}{-2pt}{-0.5pt}{0.5pt}{black!35} % {xshift}{yshift}{offset}{step}{options} 
}

%% Blind tex
	

\usepackage{blindtext}

%----------------------------------------------------------------------------------------
% Chapter wise Citations
%----------------------------------------------------------------------------------------
\usepackage[sectionbib]{natbib}
\usepackage{chapterbib}
%%% Uncomment if you change the bibliography heading/title
\renewcommand{\bibname}{References}

%%% Uncomment if you want to include the bibliographies at the end of each chapter in the table of contents.  
\usepackage[nottoc]{tocbibind}


%----------------------------------------------------------------------------------------
% Collect all your header information from the chapters here, things like acronyms, custom commands, necessary packages, etc. 
%----------------------------------------------------------------------------------------
\usepackage{parskip} %this will put spaces between paragraphs
\setlength{\parindent}{15pt} % this will create and indent on all but the first paragraph of each section. 
% should maybe change to glossaries package
\usepackage{acro}
\DeclareAcronym{est}{
	short = EST,
	long  = expressed sequence tags
}

\DeclareAcronym{Xl}{
	short = \textit{X.~laevis},
	long  = \textit{Xenopus~laevis}
}
\DeclareAcronym{Xg}{
	short = \textit{X.~gilli},
	long  = \textit{Xenopus~gilli}
}

\usepackage{etoolbox}
\preto\chapter{\acresetall} % resets acronyms for each chapter

\usepackage{xspace} %helps spacing with custom commands. 
\newcommand{\oddname}{{\sc SoME goOfY LonG ThiNg With an AwkWarD NAme}\xspace}


\usepackage{pgfplotstable} % a much better way to handle tables
\pgfplotsset{compat=1.12}





%----------------------------------------------------------------------------------------
%	THESIS INFORMATION
%----------------------------------------------------------------------------------------

%% Thesis Title (Choose only one) --------------
% \thesistitle{Big Data Clustering Models and Applications in Research and Subscription-based Platforms} % Your thesis title, print it elsewhere with \ttitle

% \thesistitle{Large-Scale Data Clustering Models with Applications in Research and Subscription-based Platforms} % Your thesis title, print it elsewhere with \ttitle

% \thesistitle{Big Data Clustering Models with Applications in Research and Subscription-based Invoicing Platforms} % Your thesis title, print it elsewhere with \ttitle

%\thesistitle{Collection of Quantum Information Notes} % Your thesis title, print it
%% -------------------------------



%\supervisor{IBM} % Your supervisor's name, print it elsewhere with \supname
%\examiner{} % Your examiner's name, print it elsewhere with \examname
%\degree {Self Study} % Your degree name, print it elsewhere with \degreename
%\shortdegree{}
%\subjectarea{Quantum Information} % Your degree area name, print it elsewhere with \area
%\author{Chris Cumming} % Your name, print it elsewhere with \authorname
%\addresses{} % Your address, print it elsewhere with \addressname
%\subject{Physics} % Your subject area, print it elsewhere with \subjectname
%\keywords{} % Keywords for your thesis, print it elsewhere with \keywordnames
%\university{\href{}} % Your university's name and URL, print it elsewhere with \univname
%\department{\href{}} % Your department's name and URL, print it elsewhere with \deptname
%\group{\href{}{}} % Your research group's name and URL, print it elsewhere with \groupname
%\faculty{\href{}{}} % Your faculty's name and URL, print it elsewhere with \facname

% this sets up hyperlinks
\hypersetup{pdftitle=\ttitle} % Set the PDF's title to your title
\hypersetup{pdfauthor=\authorname} % Set the PDF's author to your name
\hypersetup{pdfkeywords=\keywordnames} % Set the PDF's keywords to your keywords



%----------------------------------------------------------------------------------------
% Begin document
%----------------------------------------------------------------------------------------
\begin{document}
\frontmatter

\frontmatter % Use roman page numbering style (i, ii, iii, iv...) for the pre-content pages

\pagestyle{plain} % Default to the plain heading style until the thesis style is called for the body content

%----------------------------------------------------------------------------------------
%	Half Title (lay title). 
%   60 Characters with spaces
%----------------------------------------------------------------------------------------
%\begin{halftitle} % could not get this environment working
%\vspace*{\fill}
\vspace*{2.5in}
\begin{center}
\LARGE \textsc{PX3A2: Quantum Physics of Atoms}% ideally, but it doesn't seem to matter
\end{center}
%\vspace*{\fill}
\vfill
\pagenumbering{gobble} % leave this here, McMaster doesn't want this page numbered
%\end{halftitle}
%\clearpage

%----------------------------------------------------------------------------------------
%	TITLE PAGE
%----------------------------------------------------------------------------------------
\pagenumbering{gobble}
\begin{center}

\vfill
\textsc{\Large \ttitle} \\

\vspace{50pt}


\vfill
%By\\
%\uppercase\expandafter{\authorname},\\
%M.Sc. (Computer Science)\\%% -----> List prior degrees after comma  <----

\vspace{50pt}

 \vfill
{\large}\\

\vspace{30pt}

\vfill
%{\large  \degreename}\\ % \vspace{15pt}
%{\large  in}\\ %\vspace{15pt}
%{\large  \subjectareaname}\\
\vfill
\vfill

\vspace{50pt}

%{\large \univname\, \\ Hamilton, Ontario}\\[4cm] % replace \today with the submission date


%{\copyright\, Copyright by \authorname,\, \today}\\[4cm] % replace \today with the submission date

\end{center}

%----------------------------------------------------------------------------------------
%	Descriptive note numbered ii
%----------------------------------------------------------------------------------------
% Need to add below info
\newpage
\pagenumbering{roman} % leave to turn numbering back on
\setcounter{page}{2} % leave here to make this page numbered ii, a Grad School requirement

\noindent % stops indent on next line
%\degreename\, (\the\year) \\
%\deptname \\
%\univname \\ 
%Hamilton, Ontario, Canada \\[1.5cm]
%TITLE: \ttitle 
\\ \\
%AUTHOR:\\ \authorname, \\
%M.Sc. (Computer Science)\,\\  %list previous degrees  
\\ \\
%SUPERVISOR:\\ 
%\supname\,\\
%Professor, Department or School Name, \\
%McMaster University, ON, Canada
\\ 
\\
%SUPERVISORY COMMITTEE CHAIR: \\ 
%Dr. XXX XXXX,\\
%Professor, Department or School Name, \\
%McMaster University, ON, Canada
\\
\\
%SUPERVISORY COMMITTEE MEMBERS: \\Dr. XXX XXXX\\ 
%Professor, Department or School Name, \\
%McMaster University, ON, Canada
\\
\\
%Dr. XXX XXXXX\\ 
%Professor, Department or School Name, \\
%McMaster University  
\\
\\
%NUMBER OF PAGES: \pageref{lastoffront}, \pageref{LastPage}  % put in iv and number

%\clearpage



%----------------------------------------------------------------------------------------
%	ABSTRACT PAGE
%----------------------------------------------------------------------------------------
\vspace{-10cm}
\section*{\Huge Introduction} 
%\addchaptertocentry{Introduction}
% Type your abstract here. 
It should be noted that throughout this module there is a mixture of wave mechanics and matrix mechanics. You may not have realised it but so far you've dealt pretty exclusively with wave mechanics. In general in Martin's section he uses wave mechanics, but does use matrix mechanics sometimes, whereas Ben's section is just matrix mechanics. These are just different ways of performing calculations in quantum mechanics and the end result for both is equivalent, so you shouldn't worry too much about the detail behind it. All you should really worry about is that the formalism in wave mechanics is based off partial differential equations and wave theory such that the operators are derivatives acting on wave functions, and is normally used when dealing with infinite dimensional systems. Whilst matrix mechanics is much more handy when dealing with finite dimensional systems where we can use linear algebra to solve problems more easily, as the operators in this case are now represented by matrices acting on state vectors. Dirac was the one to show that these 2 formalisms are equal. He also came up with the bra-ket notation which is arguably the more fundamental way of forming problems in quantum mechanics, as it doesn't specify if we are using wave or matrix mechanics.

\noindent Another thing to note is that again so far you have only really been working in the Schrodinger picture, however there is also the Dirac/interaction picture and the Heisenberg picture, which are used in this module and later ones. These different pictures correspond to where we put the time dependence in quantum mechanics. In the Schrodinger picture, the time dependence is put in the eigenstates and the operators have no time dependence. Whilst in the Heisenberg picture we put the time dependence in the operators and the eigenstates are time independent. Like with wave mechanics and matrix mechanics these provide the same predictions as each other, its just in some cases choosing one over the other makes the maths simpler. The Dirac picture combines these together, such that some time dependence is in the operators and some in the eigenstates. 
%\clearpage
%----------------------------------------------------------------------------------------
%	ACKNOWLEDGEMENTS
%----------------------------------------------------------------------------------------

% \begin{acknowledgements}
% \addchaptertocentry{\acknowledgementname} % Add the acknowledgements to the table of contents

% The acknowledgements and the people to thank go here, don't forget to include your project adviser\ldots

% \end{acknowledgements}

%\section*{\Huge \centering Acknowledgements}
%\addchaptertocentry{Acknowledgements}
%\vspace{20pt}

\hspace{\parindent}


%\clearpage

%----------------------------------------------------------------------------------------
%	LIST OF CONTENTS/FIGURES/TABLES PAGES
%----------------------------------------------------------------------------------------

\tableofcontents % Prints the main table of contents

%\listoffigures % Prints the list of figures

%\listoftables % Prints the list of tables


%----------------------------------------------------------------------------------------
%	DECLARATION PAGE
%----------------------------------------------------------------------------------------
%\begin{declaration}
% \addchaptertocentry{\authorshipname}

%\noindent I, \authorname, declare that this thesis titled, \textbf{\ttitle}, and %works presented in it are my own. I confirm that 

%\begin{itemize} 
%\item List each chapter
%\item and what you have done for it
%\end{itemize}
 
%\end{declaration}

%----------------------------------------------------------------------------------------
% The following bit is just here to make sure we end up on a new page and get the total number of roman numeral
\label{lastoffront}
%\clearpage
% make sure this command is on the last of your frontmatter pages, i.e. only this command, a \clearpage then \mainmatter
% should be fine without modification
%----------------------------------------------------------------------------------------

%----------------------------------------------------------------------------------------
%	THESIS CONTENT - CHAPTERS
%----------------------------------------------------------------------------------------

\mainmatter % Begin numeric (1,2,3...) page numbering

\pagestyle{thesis} 

\chapter{Non-Degenerate PT}
\label{chapt1}

\section{Introduction}

The vast majority of systems we consider in quantum mechanics don't have exact solutions and so need to be approximated. In fact there are only 2 known systems which have been solved exactly. These are the simple harmonic oscillator (SHO) and the potential well in 3D. However if the system we want to find the eigenvalues and eigenstates for is similar to one of these, then we can use perturbation theory (PT) as a way to approximate these eigenvalues and eigenstates. This involves taking a system which has been solved exactly and applying a small correction to make it resemble the system we wish to solve, and then consider perturbative expansions of the values that change.

\section{Derivation}

\noindent As far as I'm aware this derivation for the boxed expressions for non-degenerate PT is not examinable, so feel free to skip it. But you might find it useful to see where the expressions come from, so I'll briefly lay it out here.

\noindent We wish to find the corrections to our energy eigenvalues and corresponding eigenstates as a result of the perturbation we have applied. We begin by writing the Hamiltonian for our new perturbed system as:

\begin{equation}
    \Hat{H} = \Hat{H}_0 + \beta \Hat{V},
    \label{ModifidedHamiltonianPT}
\end{equation}

\noindent where $\Hat{H}_0$ is the Hamiltonian of our unperturbed solved system (like the SHO), $\Hat{V}$ is the perturbation we apply and $\beta$ is just a real number that controls the size of $\Hat{V}$. Here we assume that $\Hat{H}_0$ and $\Hat{V}$ are both hermitian, so that $\Hat{H}$ is also hermitian. We further assume that the eigenvalues of the unperturbed system are not degenerate, i.e there aren't any eigenstates with the same eigenvalue or an eigenstate with multiple eigenvalues. Note $\beta$ is constrained by $0\leq \beta < 1$ otherwise the expansions will not converge.

\noindent We then consider our normal Hamiltonian eigenvalue equation:

\begin{equation}
    \Hat{H} \ket{\psi_n} = E_n \ket{\psi_n},
\end{equation}

\noindent where $\ket{\psi_n}$ is the eigenstate of the perturbed state in the $n^{th}$ energy level, and $E_n$ is the energy of the $n^{th}$ energy level and $\Hat{H}$ is given by Eq~\ref{ModifidedHamiltonianPT}.

\noindent Applying PT, the perturbative expansions of $E_n$ and $\ket{\psi_n}$ are:

\begin{equation}
    E_n = E^0_n + \beta E^1_n + \beta^2 E^2_n + \beta^3 E^3_n...,
    \label{EigenvaluePT}
\end{equation}

\begin{equation}
    \ket{\psi_n} = \ket{\phi^0_n} + \beta \ket{\phi^1_n} + \beta^2 \ket{\phi^2_n} + \beta^3 \ket{\phi^3_n}... ,
    \label{EigenstatePT}
\end{equation}

\noindent where $\ket{\phi^0_n}$ and $E^0_n$ are the eigenstates and eigenvalues of the $n^{th}$ energy level of the unperturbed system, $\Hat{H_0}$. First order and above terms in $\beta$ represent corrections to the unperturbed values as a result of the perturbation, $\Hat{V}$. The superscript corresponds to the order of the correction, so $E^1_n$, is a first order correction to the $n^{th}$ eigenvalue of the unperturbed system, $E^2_n$ a second order correction, $E^3_n$ a third order correction, etc. The same is true for the eigenstates: $\ket{\phi^1_n}$, $\ket{\phi^2_n}$ and $\ket{\phi^3_n}$. 
It might help to think of the perturbative expansions, Eq~\ref{EigenvaluePT} and Eq~\ref{EigenstatePT}, as Taylor series expansions of the eigenvalues and eigenstates about the unperturbed eigenvalue and eigenstate.

\noindent For the following we assume that unperturbed eigenstates are orthonormal to one another and that corrections to the $n^{th}$ unperturbed eigenstate are orthonormal to the unperturbed eigenstate, i.e:

\begin{equation}
    \braket{\phi^0_n | \phi^0_m} = \delta_{nm}
    \label{UnperturbedEigenstateOrthonormality}
\end{equation}

\begin{equation}
    \braket{\phi^0_n | \phi^1_n} = \braket{\phi^0_n | \phi^2_n} = \braket{\phi^0_n | \phi^3_n} = ... = 0.
    \label{EigenstateCorrectionsOrthogonal}
\end{equation}

\noindent Now combining Eq~\ref{ModifidedHamiltonianPT} - \ref{EigenstatePT}, we find:

\begin{equation}
    (\Hat{H_0} + \beta \Hat{V})(\ket{\phi^0_n} + \beta \ket{\phi^1_n} + \beta^2 \ket{\phi^2_n} + ...) = (E^0_n + \beta E^1_n + \beta^2 E^2_n + ...)(\ket{\phi^0_n} + \beta \ket{\phi^1_n} + \beta^2 \ket{\phi^2_n} + ...),
    \label{ExpandedHamiltonian}
\end{equation}

\noindent we then expand this and equate powers of $\beta$ on the left and right hand side. So for zero order terms, i.e the terms which don't contain $\beta$, we find:

\begin{equation}
    \Hat{H}_0 \ket{\phi^0_n} = E^0_n \ket{\phi^0_n},
    \label{ZeroOrderTerms}
\end{equation}

\noindent as we would expect, the unperturbed Hamiltonian operating on an unperturbed eigenstate returns that same eigenstate with the corresponding unperturbed eigenvalue. Then for first order terms:

\begin{equation}
    \Hat{H_0} \ket{\phi^1_n} + \Hat{V} \ket{\phi^0_n} = E^0_n \ket{\phi^1_n} + E^1_n \ket{\phi^0_n},
    \label{FirstOrderTerms}
\end{equation}

\noindent and to second order:

\begin{equation}
    \Hat{H_0} \ket{\phi^2_n} + \Hat{V} \ket{\phi^1_n} = E^0_n \ket{\phi^2_n} + E^1_n \ket{\phi^1_n} + E^2_n \ket{\phi^0_n}.
    \label{SecondOrderTerms}
\end{equation}

\noindent By pre-multiplying Eq~\ref{FirstOrderTerms} by $\bra{\phi^0_n}$:

\begin{equation}
    E^0_n \braket{\phi^0_n | \phi^1_n} + \braket{\phi^0_n | \Hat{V} | \phi^0_n} = E^0_n \braket{\phi^0_n | \phi^1_n} + E^1_n \braket{\phi^0_n | \phi^0_n},
    \label{FirstOrderMultiplied}
\end{equation}

\noindent where we can remove the $E^0_n$ and $E^1_n$ terms from within the bra-ket notation as they are just scalars.

\noindent This then simplifies to give us:

\begin{equation}
    \boxed{E^1_n = \braket{\phi^0_n | \Hat{V} | \phi^0_n}}
    \label{FirstOrderEnergyCorrection}
\end{equation}

\noindent thanks to the orthonormality condition, Eq~\ref{UnperturbedEigenstateOrthonormality}, we impose. This means that the first order correction to the $n^{th}$ energy eigenvalue is given by the expectation value of the $n^{th}$ energy level unperturbed eigenstates with the perturbation we have applied.

\noindent Now in order to find the first order corrections to the eigenstates, we assume it takes the form:

\begin{equation}
    \ket{\phi^1_n} = \sum_{p, p \neq n} a_{np} \ket{\phi^0_p}
    \label{AssumedFormFirstOrderEigenstateCorrection}
\end{equation}

\noindent i.e it is a linear combination of all the unperturbed eigenstates of the system, other than the one we are trying to find the correction for, $\ket{\phi^0_n}$. Then pre-multiplying Eq~\ref{FirstOrderTerms} by $\bra{\phi^0_m}$, where m $\neq n$, using Eq~\ref{AssumedFormFirstOrderEigenstateCorrection}, Eq~\ref{UnperturbedEigenstateOrthonormality} and Eq~\ref{EigenstateCorrectionsOrthogonal} we find:

\begin{equation}
    \boxed{\ket{\phi^1_n} = - \sum_{p, p \neq n} \frac{\bra{\phi^0_p}  \Hat{V}  \ket{\phi^0_n}}{{E^0_p - E^0_n}} \ket{\phi^0_p}}
    \label{FirstOrderEigenstateCorrection}
\end{equation}

\noindent Following a similar process for the second order energy correction, we find:

\begin{equation}
    \boxed{E^2_n = - \sum_{p, p \neq n} \frac{|\bra{\phi^0_p}  \Hat{V}  \ket{\phi^0_n}|^2}{{E^0_p - E^0_n}}}
    \label{SecondOrderEnergyCorrection}
\end{equation}

\noindent Note how the numerator, $|\bra{\phi^0_p}  \Hat{V}  \ket{\phi^0_n}|^2$, is always positive. So if we further assume that $E^0_p > E^0_n$ then its always true that $E^2_n < 0$.

\section{Key Results}

The important thing here is that even though we can't solve all systems exactly we can get pretty close to the actual values by considering a known system and altering it slightly with a small perturbation. The key equations, that unfortunately you have to remember and won't be given in the exam, are the boxed ones: Eq~\ref{FirstOrderEnergyCorrection}, Eq~\ref{FirstOrderEigenstateCorrection} and Eq~\ref{SecondOrderEnergyCorrection}. These give the first and second order corrections to the $n^{th}$ energy eigenvalue and the first order correction to the corresponding eigenfunction.

% remember to set these at the start of each chapter
\chapter{Degenerate PT}
\label{chapt2} 

\section{Introduction}

In the previous section we applied a perturbation to a known system and found the first and second order corrections to energy eigenvalues and the first order correction to eigenstates. However we assumed that there was no degeneracy involved. If our system we are considering does have degenerate energy levels, i.e distinct eigenstates with the same eigenvalue then problems arise. By considering Eq~\ref{FirstOrderEigenstateCorrection} and Eq~\ref{FirstOrderEnergyCorrection} we can see that if there are degenerate energy levels, where $E^0_p = E^0_n$, then these expressions diverge. Thankfully we can alter the process to account for this degeneracy and still find values for our system.

\section{Derivation}

As with the non-degenerate case, as far as I'm aware, this derivation is not examinable so feel free to skip it. Also this is not the most rigorous of derivations either.

\noindent Consider $E^0_n$, an eigenvalue of the unperturbed Hamiltonian $\Hat{H}_0$. Suppose it is s-fold degenerate, so there are s linearly independent eigenstates which have energy $E^0_n$. We shall label these as $\ket{u^0_{n \alpha}}$ and assume that they are orthornormal to each other:

\begin{equation}
        \braket{u^0_{n \alpha} | u^0_{n \beta}} = \delta_{\alpha \beta},
    \label{DegenerateEigenfunctions}
\end{equation}

\noindent where $\alpha, \beta$ = 1, 2, 3, ..., s are labels for the s degenerate eigenstates of the $n^{th}$ energy level.

\noindent Since any linear combination of $\ket{u^0_{n \alpha}}$ is an eigenstate of $\Hat{H_0}$, then the unperturbed states we used previously in non-degenerate PT are not uniquely defined and so can't be used for PT. Therefore we need to find the correct normalised states which can be used in PT. Note there will be s of these states. We write these states in the following form:

\begin{equation}
    \ket{\phi^0_{ni}} = \sum^s_{\alpha = 1} c_{i\alpha} \ket{u^0_{n \alpha}}
    \label{LinearCombinationDegenerateStates},
\end{equation}

\noindent where the $c_{i\alpha}$ ensures we have the correct linear combination of unperturbed degenerate states to use in PT, and i = 1, 2, 3, ..., s.

\noindent From Eq~\ref{FirstOrderEigenstateCorrection} and Eq~\ref{FirstOrderEnergyCorrection}, we can see the denominator tends to 0 as the system tends towards degeneracy. In order to ensure we have valid perturbative expansions, like Eq~\ref{EigenstatePT} and Eq~\ref{EigenvaluePT}, we make the numerator tend to 0 as well. This is ensured by the condition:

\begin{equation}
    \braket{\phi^0_{ni} | \Hat{V} | \phi^0_{nj}} = \braket{\phi^0_{ni} | \Hat{V} | \phi^0_{ni}} \delta_{ij}
    \label{MatrixCondition}
\end{equation}

\noindent which can be represented as an $s \times s$ diagonal matrix which ensures our perturbation expansions are valid. This means that we can now follow the same process as for the non-degenerate case, which results in a slightly modified Eq~\ref{FirstOrderEnergyCorrection}:

\begin{equation}
    \boxed{E^1_{ni} = \braket{\phi^0_{ni} | \Hat{V} | \phi^0_{ni}}}.
    \label{FirstOrderEnergyCorrectionsDegenerate}
\end{equation}

\noindent Analogous to before, the first order correction for the $i^{th}$ degenerate state of the $n^{th}$ energy level is given by the expectation value of the $i^{th}$ degenerate eigenstate with the perturbation.

\noindent However you wouldn't normally used Eq~\ref{FirstOrderEnergyCorrectionsDegenerate}, since it doesn't tell you the correct linear combination of $\ket{u^0_{n \alpha}}$ which allows PT to work. There is an easier way to find the first order corrections for each of the degenerate eigenstates as well as the coefficients, $c_{i \alpha}$. You'll have to forgive me for not writing out the derivation for this as its just disgusting, but results in solving the eigenvalue equation:

\begin{equation}
    \boxed{\sum^s_{\alpha = 1} (\braket{u^0_{n\beta} | \Hat{V} | u^0_{n\alpha}} - E^1_{ni} )c_{i \alpha} = 0}.
    \label{DegenerateMatrixEquation}
\end{equation}

\noindent This is most nicely represented by a matrix equation, where $\sum^s_{\alpha = 1} (\braket{u^0_{n\beta} | \Hat{V} | u^0_{n\alpha}}$ is the matrix, the elements of which are determined by the expectation value of the degenerate eigenstates with the perturbation. This then operates on the eigenstate with components $c_{i \alpha}$, and then returns the same eigenstate multiplied by the eigenvalue of $E^1_{ni}$.

\noindent So in order to find the first order corrections to the degenerate energy levels and their corresponding wave functions you simply solve Eq~\ref{DegenerateMatrixEquation} like any other eigenvalue equation. The eigenvalues of the matrix determine the first order corrections, $E^1_{ni}$, which are then re-inserted back into the equation to find $c_{i \alpha}$. This then means you can determine $\ket{\phi_{ni}}$ for each $E^1_{ni}$, using Eq~\ref{LinearCombinationDegenerateStates}.

\noindent Don't worry if none of this is making any sense it will hopefully make a lot more sense once you see an example and there aren't tons of greek letters and summation symbols all over the place.


\section{Example: 2D SHO}

In 1D the SHO is not a degenerate system, it has unique eigenstates for each of the energy levels given by $E_n = (n + \frac{1}{2})\hbar \omega$. However when we consider a 2D SHO, we are allowing oscillations in two directions, so the unperturbed Hamiltonian is just the sum of the 1D Hamiltonian in two directions:

\begin{equation}
    \Hat{H_0} = \frac{p^2_x}{2m} + \frac{p^2_y}{2m} + \frac{1}{2} m \omega^2 (x^2 + y^2),
    \label{UnperturbedHamiltonianSHO}
\end{equation}

\noindent where $p_x$, $p_y$ are the momentum in the $x$ and $y$ directions respectively, and $x$, $y$ the canonical coordinates. The unperturbed energy of this 2D system will now have contributions from the springs in both the $x$ and $y$ directions:

\begin{equation}
    E^0_N = N \hbar \omega = (n_x + n_y + 1) \hbar \omega = [(n_x + \frac{1}{2}) + (n_y + \frac{1}{2})] \hbar \omega,
    \label{UnperturbedEnergy2DSHO}
\end{equation}

\noindent with eigenstates given by:

\begin{equation}
    u^0_N(x, y) = u^0_{n_x}(x)u^0_{n_y}(y),
    \label{UnperturbedEigenstates2DSHO}
\end{equation}

\noindent where $u^0_{n_x}(x)$ are just the eigenstates for the 1D system.

\noindent Now when $N$ = 1, we can see there is no degeneracy present as in this case the only possibility is $n_x = n_y = 0$. So for $E^0_1$ the only possible eigenstate is $u^0_0 (x, y) = u^0_0(x)u^0_0(y)$. However if we consider the first excited state $N$ = 2, there is a 2-fold degeneracy, so $s$ = 2. If $N$ = 2, then there are 2 sets of possible values of $n_x, n_y$. These are $n_x = 0$ and $n_y = 1$, or $n_x = 1$ and $n_y = 0$. Both of these sets would give $E^0_2 = 2\hbar \omega$. In the first set the spring in the $x$ direction isn't oscillating it only has its zero point energy, whilst in the $y$ direction the spring is oscillating in its first excited state.

\noindent This means our degenerate eigenstates, using Eq~\ref{UnperturbedEigenstates2DSHO}, are given by:

\begin{eqnarray}
    u^0_{21}(x, y) = u^0_1(x)u^0_0(y) \\
    u^0_{22}(x, y) = u^0_0(x)u^0_1(y),
    \label{DegenerateWavefunctions2DSHO}
\end{eqnarray}

\noindent where the first equation corresponds to $n_x = 1$, $n_y = 0$ and the second corresponds to $n_x = 0$, $n_y = 1$. These equations correspond to our $\ket{u^0_{n\alpha}}$, defined by Eq~\ref{DegenerateEigenfunctions}, where $N \equiv n = 2$ as we are in the first excited state of the system, and since there are 2 degenerate eigenstates $\alpha = 1, 2$. Now using Eq~\ref{LinearCombinationDegenerateStates}, we define the correct linear combination of degenerate eigenstates that allows us to perform PT:

\begin{eqnarray}
    \ket{\phi^0_{21}} = c_{11}\ket{u^0_{21}} + c_{12}\ket{u^0_{22}} \label{LinearCombinationDegenerateStates2DSHO_1} \\
    \ket{\phi^0_{22}} = c_{21}\ket{u^0_{21}} + c_{22}\ket{u^0_{22}},
    \label{LinearCombinationDegenerateStates2DSHO_2}
\end{eqnarray}

\noindent these are our $\ket{\phi^0_{ni}}$ and $c_{i\alpha}$.

\noindent Now in order to find the first order corrections and correct linear combination, we need to solve Eq~\ref{DegenerateMatrixEquation}. The matrix will be a $2 \times 2$ matrix since this a 2-fold degeneracy. The elements of the matrix are determined as:

\begin{eqnarray}
    V_{11} = \braket{u^0_{21} | \Hat{V} | u^0_{21}}  \\
    V_{12} = \braket{u^0_{21} | \Hat{V} | u^0_{22}} \\
    V_{21} = \braket{u^0_{22} | \Hat{V} | u^0_{21}} \\
    V_{22} = \braket{u^0_{22} | \Hat{V} | u^0_{22}},
    \label{MatrixElements2DSHO}
\end{eqnarray}

\noindent where $V_{ij}$ is the matrix element of the $i^{th}$ row and the $j^{th}$ column. So the perturbation matrix looks like:
\begin{equation}
    \Hat{V} \doteq \begin{pmatrix} V_{11} & V_{12} \\ V_{21} & V_{22} \end{pmatrix}
    \label{PerturbationMatrix2DSHO}
\end{equation}

\noindent Now assuming a perturbation of $\Hat{V}$ = $\lambda m  \omega^2 xy$, and standard forms for the unperturbed eigenstates of the 1D SHO, we find by evaluating the matrix elements, that Eq~\ref{PerturbationMatrix2DSHO} is given by:

\begin{equation}
    \Hat{V} \doteq \begin{pmatrix} 0 & \frac{\lambda \hbar \omega}{2} \\ \frac{\lambda \hbar \omega}{2} & 0 \end{pmatrix}
\end{equation}

\noindent The first order corrections to the first excited energy level are then found from the eigenvalues of the above matrix to be $E^1_2 = \pm \frac{\lambda \hbar \omega}{2}$. Then our coefficients in Eq~\ref{LinearCombinationDegenerateStates2DSHO_1} and Eq~\ref{LinearCombinationDegenerateStates2DSHO_2} are determined from:

\begin{equation}
    \begin{pmatrix} 0 & \frac{\lambda \hbar \omega}{2} \\ \frac{\lambda \hbar \omega}{2} & 0 \end{pmatrix} \begin{pmatrix} c_{11} \\ c_{12} \end{pmatrix} = + \frac{\lambda \hbar \omega}{2} \begin{pmatrix} c_{11} \\ c_{12} \end{pmatrix}
\end{equation}

\noindent for Eq~\ref{LinearCombinationDegenerateStates2DSHO_1}, and:

\begin{equation}
    \begin{pmatrix} 0 & \frac{\lambda \hbar \omega}{2} \\ \frac{\lambda \hbar \omega}{2} & 0 \end{pmatrix} \begin{pmatrix} c_{21} \\ c_{22} \end{pmatrix} = - \frac{\lambda \hbar \omega}{2} \begin{pmatrix} c_{21} \\ c_{22} \end{pmatrix}
\end{equation}

\noindent for Eq~\ref{LinearCombinationDegenerateStates2DSHO_2}.

\section{Key Results}
Degenerate PT allows us to find approximate values for the energy eigenvalues and eigenstates of a degenerate system which resembles a system we can solve exactly. If you get asked to find the first order energy corrections for a degenerate system and/or the correct eigenstates that allow PT to work, then the process to follow is:

\begin{itemize}
    \item Check what the degeneracy of the system is ($s$).
    \item Find the forms of the $s$ degenerate eigenstates, Eq~\ref{DegenerateEigenfunctions}.
    \item Form $s$ linear combinations from the $s$ degenerate eigenstates, Eq~\ref{LinearCombinationDegenerateStates}.
    \item Form the $s \times s$ perturbation matrix using the perturbation given and the degenerate eigenstates. Since the matrix is symmetric you won't have to calculate all the off diagonal elements individually, Eq~\ref{DegenerateMatrixEquation}.
    \item Calculate the eigenvalues of the matrix, these give you the first order energy corrections of the degenerate energy level you're considering. Note there will be $s$ of them, but some may be repeated.
    \item Then form an eigenvalue equation, with the perturbation matrix, the linear combinations of degenerate eigenstates and the eigenvalues to find the correct linear combinations.
\end{itemize}

\noindent Note the method outlined in this section is a brute force method. There is a more 'elegant' method, which involves symmetry arguments that Martin goes through in Appendix B of chapter 9 of his notes. I'll briefly outline this method in Section \ref{chapt4}.

\chapter{Variational Method}
\label{chapt3}

\section{Introduction}

The variational, or Rayleigh-Ritz, method is another approximation method we can use. We use this when dealing with a system where only the form of the Hamiltonian is known exactly. It involves making a prediction of what the ground state wave function of the system would look like and using this to calculate an upper bound for the energy of the ground state. This method can be applied to excited states as well, however it is slightly more complicated and you don't need to worry about this.

\section{Derivation}

From the third postulate of quantum mechanics, we know that any state of a particular system, $\psi$, can be expressed as a linear combination of the eigenstates of an operator, such as the Hamiltonian, so long as they form a complete set:

\begin{equation}
    \ket{\psi} = \sum_n = c_n \ket{\phi_n},
    \label{ExpansionWavefunction}
\end{equation}

\noindent where $\ket{\phi_n}$ are the eigenstates of $\Hat{H}$ and the sum is over the complete set of these eigenstates. Then considering the expectation value of $\Hat{H}$:

\begin{equation}
    \braket{\Hat{H}} = \frac{\braket{\psi | \Hat{H} | \psi}}{\braket{\psi | \psi}} = \braket{\psi | \Hat{H} | \psi},
    \label{ExpectationValue}
\end{equation}

\noindent where we assume $\psi$ is normalised. Then inserting Eq~\ref{ExpansionWavefunction} into Eq~\ref{ExpectationValue}, we find:

\begin{equation}
    \braket{\Hat{H}} = \sum_n = |c_n|^2 E_n.
    \label{ExpectationValueExpanded}
\end{equation}

\noindent This means if we take $\psi$ to be our trial wave function that we guess, which I'll label as $\psi_T$, then $\braket{\Hat{H}} \geq E_1$ which is the ground state of the system. This is because we assume that the energy eigenvalues, $E_n$, of $\phi_n$ are ordered such that $E_1 < E_2 < E_3 < E_4$. So unless we accidentally pick the exact ground eigenstate of the Hamiltonian then our trial wave function will be a combination of excited eigenstates of the Hamiltonian, as we predict from Eq~\ref{ExpansionWavefunction}. So our expectation value of the Hamiltonian will contain the sum of the corresponding energy eigenvalues for these eigenstates. For example if we let our trial wave function take the form:

\begin{equation}
    \ket{\psi_T} = c_1 \ket{\phi_1} + c_5 \ket{\phi_5} + c_{23} \ket{\phi_{22}},
    \label{Example1Wavefunction}
\end{equation}

\noindent where $c_1, c_5, c_{23} \neq 0$. Then Eq~\ref{ExpectationValueExpanded} predicts:

\begin{equation}
    \braket{\Hat{H}} = |c_1|^2 E_1 + |c_5|^2 E_5 + |c_{23}|^2 E_{23},
    \label{Example1Value}
\end{equation}

\noindent which must be larger than $E_1$. However if we say that:

\begin{equation}
    \ket{\psi_T} = c_1 \ket{\phi_1} = \ket{\phi_1},
    \label{Example2Wavefunction}
\end{equation}

\noindent where $c_1 = 1$ as $\ket{\psi_T}$ must be normalised. Then from Eq~\ref{ExpectationValueExpanded}:

\begin{equation}
    \braket{\Hat{H}} = |c_1|^2 E_1 = E_1.
    \label{Example2Value}
\end{equation}

\noindent We choose a form for $\psi_T$ which is a function of some parameters that we can tune, $\alpha_i$, i.e $\psi_T = \psi_T (\alpha_1, \alpha_2, \alpha_3, ...)$. We then calculate the expectation value of the Hamiltonian with $\psi_T$ using Eq~\ref{ExpectationValue}. The expectation value is then minimised with respect to $\alpha_1, \alpha_2, \alpha_3, ...$ with the final result representing the best estimate of the ground state energy of the system using $\psi_T$ and an upper bound for the actual ground state energy of the system.


\section{Key Results}

The general process for the variational method to always follow is:

\begin{itemize}
    \item Ensure that $\psi_T$ is normalised.
    \item Calculate $\braket{\Hat{H}}$ using trial wave function.
    \item Differentiate $\braket{\Hat{H}}$ with respect to the variational parameter.
    \item Calculate minimum value of $\braket{\Hat{H}}$ using previously found value for variational parameter.
\end{itemize}

\noindent In an exam you will never have to guess a wave function yourself, it will always be given to you and there will always only be one parameter to minimise $\braket{\Hat{H}}$ with respect to. Note if we have 2 trial wave functions, and one of them gives a lower value for $\braket{\Hat{H}}$ than the other, then this trial wave function gives a better estimate for the actual ground state energy of the system.

\section{Exam Question?}

\chapter{SOC and the Zeeman effect}
\label{chapt4}

\section{Introduction}

This entire section is effectively a massive example of how useful degenerate PT is. Here we apply it to a hydrogenic system to quantify the magnitude of energy splittings when we apply an external magnetic field, allow spin-orbit coupling (SOC) or have a combination of both. This is done both using the brute force method, described in Section~\ref{chapt2}, and by figuring out what the correct eigenstates for PT are using symmetry, which I'll explain briefly in a minute.

\noindent Unfortunately the notation isn't great in places here, because in some cases V means the scalar potential of the nucleus and in others it means the potential energy of the electron in the scalar potential of the nucleus. I've tried to make it clear here using $V_c$ meaning the potential energy. There's also some funny business going on with operator symbols.

\section{Derivation}

The Zeeman effect involves an external magnetic field coupling to the magnetic moments of an electron and SOC involves the coupling between these magnetic moments. In general the magnetic moment of a particle due to orbital motion is:

\begin{equation}
    \underline{\mu}_L = g_L \frac{q}{2m} \underline{L},
    \label{OrbitalMagneticMoment}
\end{equation}

\noindent where $g_L$ is the orbital gyromagnetic ration, $q$ is the charge of the particle, $m$ the mass and $\underline{L}$ the orbital angular momentum vector. However all particles also have intrinisc angular momentum (spin), which also gives rise to a magnetic moment:

\begin{equation}
    \underline{\mu}_S = g_S \frac{q}{2m} \underline{S},
    \label{SpinMagneticMoment}
\end{equation}

\noindent where $g_S$ is the spin gyromagnetic ratio and $\underline{S}$ the spin angular momentum vector.

\noindent For an electron $g_L = 1$, $g_S = 2$ and $q = -e$. Note that $g_S$ doesn't exactly equal 2 there are some corrections to this value due to QED, which is also a perturbation theory but in the context of quantum field theory. 

\noindent Now if we are going to apply degenerate PT to a hydrogenic system that experiences the Zeeman effect and SOC, we need to know the forms of the perturbation, $\Hat{V}$, we apply. For the Zeeman effect, the operator is found from the interaction energy between $\underline\mu$ and $\underline{B}$:

\begin{equation}
    \Hat{V}_{mag} = - \underline{\Hat{\mu}} \cdot \underline{\Hat{B}} = \frac{e}{2m}(\Hat{L}_z + 2\Hat{S}_z) \cdot B,
    \label{ZeemanEffectPerturbation}
\end{equation}

\noindent where we assume that $\underline{B} = (0, 0, B)$, i.e we align the magnetic field along the $\Hat{z}$ direction, and $\Hat{L}_z$ and $\Hat{S}_z$ are the normal operators for returning the z component of the orbital and spin angular momentum.

\noindent Unfortunately the operator for the SOC perturbation is a tad more complicated. In the rest frame of the electron, the nucleus is orbiting it. This means the electron "sees" a magnetic field due do the motion of the charged nucleus. It is given by:

\begin{equation}
    \underline{B} = \frac{-\underline{v} \times \underline{E}(r)}{c^2},
    \label{MagneticFieldSOC}
\end{equation}

\noindent where $\underline{v}$ is the relative velocity of the nucleus, $\underline{E}(r)$ is the electric field due to the nucleus and $c$ is the speed of light. Since the electric field of the nucleus is spherically symmetric we may write it as:

\begin{equation}
    \underline{E}(r) = -\underline{\nabla} V = -\frac{\partial V}{\partial r} \Hat{\underline{r}},
    \label{SphericalElectricField}
\end{equation}

\noindent where $V$ is the scalar electric potential of the nucleus. The operator for SOC takes the form:

\begin{equation}
    \Hat{V}_{SOC} = - \frac{1}{2} \underline{\Hat{\mu}}_s \cdot \underline{\Hat{B}},
    \label{SOCPerturbation}
\end{equation}

\noindent where the factor of $\frac{1}{2}$ arises from the Thomas precession which is a relativistic effect. So substituting in Eq~\ref{SpinMagneticMoment} and Eq~\ref{MagneticFieldSOC} into the above equation we find:

\begin{equation}
    \Hat{V}_{SOC} = \frac{1}{2} \frac{e}{m} \frac{\underline{\Hat{S}} \cdot (\underline{v} \times \underline{r})}{c^2} \frac{1}{r} \frac{\partial V}{\partial r},
\end{equation}

\noindent where we have used that the radial unit vector, $\underline{\Hat{r}} = \frac{\underline{r}}{r}$. Then using the relations $\underline{p} = m \underline{v}$, $\underline{\Hat{L}} = \underline{\Hat{r}} \times \underline{p} = - \underline{p} \times \underline{\Hat{r}}$ and $\underline{\Hat{S}} \cdot \underline{\Hat{L}} = \underline{\Hat{L}} \cdot \underline{\Hat{S}}$, we find:

\begin{equation}
    \frac{-e}{2m^2c^2} \frac{1}{r} \frac{\partial V}{\partial r} \underline{\Hat{L}} \cdot \underline{\Hat{S}}.
    \label{SOCPerturbationForm1}
\end{equation}

\noindent There are a number of other forms that this operator can take, the most useful of which is found using:

\begin{equation}
    \Hat{J}^2 = (\Hat{L} + \Hat{S})^2 = \Hat{L}^2 + \Hat{S}^2 + 2 \Hat{L} \cdot \Hat{S},
    \label{TotalAngularMomentumExpansion}
\end{equation}

\noindent in combination with Eq~\ref{SOCPerturbationForm1}:

\begin{equation}
    \Hat{V}_{SOC} = f(r)[j(j + 1) - l(l + 1) - s(s + 1)],
    \label{SOCPerturbationForm2}
\end{equation}

\noindent SHOULD PROBABLY CHANGE THIS AVOID CONFUSION AS IS TECHNICALLY WRONG. where it should be noted that the square brackets here don't mean commutator they are just brackets, and $f(r)$ is given by:

\begin{equation}
    f(r) = \frac{\hbar^2}{4m^2c^2} \frac{1}{r} \frac{\partial V_c}{\partial r},
    \label{f(r)Form}
\end{equation}

\noindent where $V_c(r)$ is the potential energy of the electron in the nucleus' electric field, $\frac{-Ze^2}{4\pi\epsilon_0 r}$.

\subsection{Correct eigenstates using symmetry}

As I said at the end of Section~\ref{chapt2}, there is a more elegant way of doing degenerate PT. Instead of forming a matrix and finding its eigenvalues and corresponding eigenstates, we can figure out the correct eigenstates to use for degenerate PT using symmetry arguments.

\noindent The correct eigenstates for degenerate PT are ones that commute with the unperturbed hamiltonian, $\Hat{H}_0$, the perturbation we apply, $\Hat{V}$, and an arbitrary operator, $\Hat{A}$, that we are free to choose. Once these eigenstates have been identified you can just use Eq~\ref{FirstOrderEnergyCorrectionsDegenerate} to calculate the first order energy corrections for a degenerate system. Remember this expression is just the equivalent expression for Eq~\ref{FirstOrderEnergyCorrection} that accounts for degeneracy.

\noindent The reason this works is because any operators that commute with one another must share a set of common eigenstates. So if $[\Hat{H}_0, \Hat{A}] = [\Hat{V}, \Hat{A}] = 0$, then they all share a common set of eigenstates. Then so long as all the eigenvalues of $\Hat{A}$ are unique, this automatically leads to the diagonalisation condition, Eq~\ref{MatrixCondition}, being satisfied. This in turn means that those eigenstates are the correct unperturbed states for degenerate PT.

\noindent I wouldn't worry too much about trying to understand why this works, just know that the correct eigenstates for degenerate PT are ones that are eigenstates of $\Hat{H}_0$, $\Hat{V}$ and another operator, $\Hat{A}$.

\subsection{Spin-Orbit Coupling}

We'll start off by considering the case of just SOC, so no external magnetic field has been applied. I'll start off with the symmetry argument method. Since we are considering a hydrogenic system, i.e a single electron orbiting a nucleus, the eigenstates of the unperturbed system only depend on the principal quantum number, $n$. We will use Eq~\ref{SOCPerturbationForm2} as our perturbation, which requires eigenstates that depend on $l, s$ and $j$. Now we choose our other operator, $\Hat{A}$, to be $\Hat{J}_z$. This is just the total angular momentum equivalent of $\Hat{L}_z$ or $\Hat{S}_z$, i.e $\Hat{J}_z = \Hat{L}_z + \Hat{S}_z$. This means the correct eigenstates for degenerate PT look like $\ket{n, l, s, j, m_j}$. So now we can figure out the first order energy corrections using these eigenstates and Eq~\ref{FirstOrderEnergyCorrectionsDegenerate}:

\begin{equation}
    E^1_{SOC} = \braket{n, l, s, j, m_j | \Hat{V}_{SOC} | n, l, s, j, m_j},
    \label{FirstOrderEnergySOC}
\end{equation}

\noindent then inserting Eq~\ref{SOCPerturbationForm2} and Eq~\ref{f(r)Form}, we find:

\begin{equation}
    E^1_{SOC} = \frac{\hbar^2}{4m^2_ec^2}[j(j + 1) -l(l + 1) - s(s + 1)] \braket{\frac{1}{r}\frac{\partial V_c}{\partial r}},
    \label{FirstOrderEnergySOCForm1}
\end{equation}

\noindent where again the square brackets are just brackets and don't mean commutator, and the expectation value of $\frac{1}{r} \frac{\partial V_c}{\partial r}$ depends on which $(n,l)$ orbital the electron occupies. You won't be asked to calculate what this expectation value is since its quite hellish to derive, but simplifying the above expression we find:

\begin{equation}
    E^1_{SOC} = \frac{|E_n| \alpha^2}{n}(\frac{1}{l + \frac{1}{2}} - \frac{1}{j + \frac{1}{2}})
    \label{FirstOrderEnergySOCForm2}
\end{equation}

\noindent where $\alpha$ is the fine structure constant which equals $\frac{1}{137}$ and $E_n$ are the energy levels of a hydrogenic system, Eq~\ref{HydrogenicAtomEnergyLevels}. Normally the factor in front is just written as $\lambda$.

\noindent From this equation we see that if we are in the $s$ orbital of an energy level, i.e $l = 0$, then $E^1_{SOC} = 0$ since $j = s = \frac{1}{2}$, meaning the energy levels remain the same. However if $l \neq 0$ then each orbital is split into 2 levels, given by $j = l \pm \frac{1}{2}$, with each level being $2j + 1$ degenerate.

\noindent For the $p$ orbital, $l = 1$, then $j$ may either be $\frac{3}{2}$ or $\frac{1}{2}$, with the corresponding corrections being $E^1_{SOC} = \frac{\lambda}{6}$ and $E^1_{SOC} = \frac{-\lambda}{2}$. INSERT FIGURE.

\noindent This problem can also be solved using the brute force method, here the degenerate eigenstates are described by $\ket{m_l, m_s}$. This time we use a different form of the perturbation for SOC, we take Eq~\ref{SOCPerturbationForm1} and expand out $\underline{\Hat{L}} \cdot \underline{\Hat{S}}$ in terms of the raising and lowering operators, so that:

\begin{equation}
    \Hat{V}_{SOC} = \delta \underline{\Hat{L}} \cdot \underline{\Hat{S}} = \delta [\Hat{L}_z \Hat{S}_z + \frac{1}{2}(\Hat{L}_+ \Hat{S}_- + \Hat{L}_- \Hat{S}_+)],
    \label{PerturbationSOCFormMatrix}
\end{equation}

\noindent where again the square brackets don't mean commutator. As we described in Section~\ref{chapt2}, we form the perturbation matrix using our degenerate eigenstates $\ket{m_l, m_s}$ and the perturbation given by Eq~\ref{PerturbationSOCFormMatrix}. At this point its very important to remember that the degenerate eigenstates are orthonormal to each other, Eq~\ref{DegenerateEigenfunctions}, as well as the effect that the raising and lowering operators have on eigenstates. Note that the raising and lowering operators will be given to you in the exam.

\noindent Then calculating the eigenvalues of this matrix, which correspond to the first order energy corrections, we find that we get $E^1_{SOC} = \frac{\delta}{2}$ four times and $E^1_{SOC} = -\delta$ two times. This result matches up with that found using the correct unperturbed eigenstates, as we would expect.

\subsection{Strong Field Zeeman Effect}

If we apply an external magnetic field to a system, we will get the Zeeman effect. If this applied field is larger relative to the SOC field, i.e $\Hat{V}_{mag} \gg \Hat{V}_{SOC}$, then we are in the regime of the Strong field Zeeman effect. This means we can neglect any effects due to SOC and consider our unperturbed system as just the hydrogenic system again. In this case we use the neat method to find the first order energy corrections. The correct unperturbed eigenstates are given by $\ket{n, l, m_l, m_s}$, with our perturbation described by Eq~\ref{ZeemanEffectPerturbation}, so  the first order energy corrections are found using Eq~\ref{FirstOrderEnergyCorrectionsDegenerate} to be:

\begin{equation}
    E^1_{mag} = \mu_B B_z (m_l + 2m_s),
    \label{ZeemanEffectEnergyCorrections}
\end{equation}

\noindent where $\mu_B$ is the Bohr Magneton, given by $\frac{e\hbar}{2m_e}$. Note this is also known as the Paschen-Back effect. 

INSERT FIGURE HERE

\subsection{Weak Field Zeeman Effect}

Finally we consider what happens when we apply a small external magnetic field such that $\Hat{V}_{SOC} \gg \Hat{V}_{mag}$. This means that our unperturbed system includes SOC, i.e the unperturbed Hamiltonian is $\Hat{H}_0 + \Hat{V}_{SOC}$, with the external magnetic field acting as the perturbation. This problem again can be solved either with the brute force method or using the correct unperturbed eigenstates directly.

\noindent The correct unperturbed eigenstates are, as before, $\ket{n, l, s, j, m_j}$ with the perturbation given by Eq~\ref{ZeemanEffectPerturbation}, so using Eq~\ref{FirstOrderEnergyCorrectionsDegenerate} and that $\Hat{L}_z + 2\Hat{S}_z = \Hat{L}_z + \Hat{S}_z + \Hat{S}_z = \Hat{J}_z + \Hat{S}_z$, we find that the first order energy corrections take the form:

\begin{equation}
    E^1_{mag} = m_j g_L \mu_B B,
    \label{WeakFieldEnergyCorrections}
\end{equation}

\noindent where $g_L$ is the Lande $g$ factor:

\begin{equation}
    g_L = 1 + \frac{j (j + 1) + s (s + 1) - l (l + 1)}{2j(j + 1)}
    \label{LandGFactor}.
\end{equation}

\noindent This means that each energy level is split into $2j + 1$ energy levels separated by $g_L \mu_B B$.

\noindent Alternatively we can solve this using the brute force method. Here the degenerate eigenstates are once again $\ket{m_l, m_s}$, and we use the unmodified version of Eq~\ref{ZeemanEffectPerturbation} as our perturbation. Here we find all the corresponding perturbation matrix elements and add them to the original matrix we had when considering just SOC. Then as per usual we find the eigenvalues of this matrix to find the first order energy corrections, which are found to be: $\frac{\delta}{2} + 2\mu_B B_z$, $\frac{\delta}{2} - 2\mu_B B_z$, $\frac{\delta}{2} + \frac{2}{3}\mu_B B_z$, $-\delta + \frac{1}{3}\mu_B B_z$, $\frac{\delta}{2} - \frac{2}{3}\mu_B B_z$ and $-\delta - \frac{1}{3}\mu_B B_z$ for an $l = 1, s = \frac{1}{2}$ system.


\noindent Can be very useful to split up matrices into smaller matrices and then find the determinants of these, as a lot of terms will be 0, so can consider 1x1, 2x2, etc determinants.

\section{Key Results}





\section{Exam Question?}



\chapter{Identical Particles}
\label{chapt5}


\section{Introduction}

It is impossible to follow and identify one particular particle when dealing with a quantum system of identical particles. This is because any measurement we wish to make of the system will disturb it and all the particles are indistinguishable from one another.

\noindent The wave function for a system which contains N indistinguishable particles is $\psi(\underline{r}_1, \underline{r}_2, ..., \underline{r}_k, ..., \underline{r}_N, t)$, where $\underline{r}_k$ is the position vector for the $k^{th}$ particle. This has the same normalisation condition as for a single particle state:

\begin{equation}
    \braket{\psi | \psi} = 1 = \int | \psi(\underline{r}_1, \underline{r}_2, ..., \underline{r}_k, ..., \underline{r}_N, t) |^2 d^3\underline{r}_1 d^3\underline{r}_2 ... d^3\underline{r}_k ... d^3\underline{r}_N
    \label{ManyParticleNormalisation}
\end{equation}

\noindent This says that at some time t, there is a particle within $d^3\underline{r}_1$ of $\underline{r}_1$ and another particle within $d^3\underline{r}_2$ of $\underline{r}_2$, etc and not that there is a specific particle, since we can't distinguish between them.

\noindent Operators acting on multi-particle eigenstates must be symmetric under the permutation of particle labels, due to this indistinguishability. An example of this is the Hamiltonian for the two electrons orbiting a Helium nucleus:

\begin{equation}
    \Hat{H}(1, 2) = \sum_{i=1}^2 (\frac{-\hbar^2}{2m}\nabla^2_i - \frac{2e^2}{4 \pi \epsilon_0 r_i}) + \frac{e^2}{4 \pi \epsilon_0 r_{12}}
    \label{HeliumAtomHamiltonian}
\end{equation}

\noindent The indistinguishability of the electrons means that:

\begin{equation}
    \Hat{H}(1, 2) = \Hat{H}(2, 1)
    \label{ExchangeDegeneracy}
\end{equation}

\noindent as no matter the order of the labels in the Hamiltonian we should get the same energy value. This means that the eigenvalues are degenerate with respect to these operators, this is known as exchange degeneracy.

\section{Particle Exchange Operator}

We can define an operator, $\Hat{P}_{ij}$ that exchanges the $i^{th}$ and $j^{th}$ particles with each other, such that:

\begin{eqnarray}
    \Hat{P}_{ij} \Hat{H}(i, j) \ket{\psi(i, j)} = \Hat{H}(j, i) \ket{\psi(j, i)} \\
    \hspace{15mm} =\Hat{H}(i, j) \Hat{P}_{ij} \ket{\psi(i, j)},
\end{eqnarray}

\noindent where in between the lines we use Eq~\ref{ExchangeDegeneracy} and that $\ket{\psi(j, i)} = \Hat{P}_{ij} \ket{\psi(i, j)}$. From this we can see that:

\begin{equation}
    \Hat{P}_{ij} \Hat{H}(i, j) \ket{\psi(i, j)} = \Hat{H}(i, j) \Hat{P}_{ij} \ket{\psi(i, j)},
    \label{ExchangeOperatorCommutator}
\end{equation}

\noindent which means that $[\Hat{H}(i, j), \Hat{P}_{ij}] = 0$, i.e the exchange operator commutes with the Hamiltonian operator, meaning they share a common set of eignenstates. Also as particles are indistinguishable, $\Hat{P}_{ij}$ must also commute with any operator that corresponds to an observable and so also share a common set of eigenstates.

\noindent Applying the exchange operator twice on the same state returns us to the original state:

\begin{equation}
    \Hat{P}_{ij} \Hat{P}_{ij} \ket{\psi(i, j)} = \Hat{P}_{ij} \ket{\psi(j, i)} = \ket{\psi(i, j)},
\end{equation}

\noindent this means that the eigenvalues of $\Hat{P}_{ij}$ must be $\pm1$. The $+1$ eigenvalue indicates a symmetric eigenstate, whilst the $-1$ eigenvalue indicates a fully anti-symmetric eigenstate. This means any physically acceptable eigenstate that represents identical particles must be symmetric or fully anti-symmetric due to $\Hat{P}_{ij}$ commutating with operators representing physical observables and hence share a common set of eigenstates. Note that due to the TDSE, a symmetric eigenstate will remain symmetric and the same is true for an anti symmetric eigenstate.

\section{Fermions and Bosons}

If our eigenstate describes a collection of identical particles with half integer spin, then it is anti-symmetric, for example:

\begin{equation}
    \Psi (1, 2) = \frac{1}{\sqrt{2}} (\psi_a(1) \psi_b (2) - \psi_a (2) \psi_b (1)),
    \label{AntiSymmetricEigenfunction}
\end{equation}

\noindent and so when the exchange operator is applied, we get an eigenvalue of $-1$. Note how if one of the particles (either particle  1 or 2) occupies the same quantum state as the other (a = b), then the eigenstate disappears:

\begin{equation}
    \Psi (1, 2) = \frac{1}{\sqrt{2}} (\psi_a(1) \psi_a (2) - \psi_a (2) \psi_a (1)) = 0.
\end{equation}

\noindent This is the Pauli Exclusion Principle (PEP). It says that any two half integer spin particles cannot occupy the exact same quantum state as each other. These half integer spin particles are known as Fermions, and they obey Fermi-Dirac statistics which is given by the Fermi-Dirac function:

\begin{equation}
    f(E) = \frac{1}{e^{\beta (E - \mu)} + 1},
    \label{FermiDiracDistributionFunction}
\end{equation}

\noindent where $f(E)$ gives the average occupation of a quantum state of energy E, $\beta = \frac{1}{k_B T}$ and $\mu$ is the chemical potential of the system.

\noindent However if our eigenstate describes a collection of identical particles with integer spin, then it is symmetric:

\begin{equation}
    \Psi (1, 2) = \frac{1}{\sqrt{2}} (\psi_a(1) \psi_b (2) + \psi_a (2) \psi_b (1)),
    \label{SymmetricEigenfunction}
\end{equation}

\noindent where upon application of the exchange operator we the an eigenvalue of $+1$. These particles have no such restriction like the PEP. Integer spin particles are known as bosons and obey Bose-Einstein statistics which is described by the Bose-Einstein function:

\begin{equation}
    f(E) = \frac{1}{e^{\beta (E - \mu)} - 1},
    \label{BoseEinsteinDistributionFunction}
\end{equation}

\noindent where the only difference from Eq~\ref{FermiDiracDistributionFunction} is that the sign in front of the 1 has changed.

\noindent Normally eigenstates, at least of simple systems, can be split into a spatial component and a spin component. If we are considering a fermion, then this eigenstate must be anti-symmetric, this can either be achieved with either the spatial or spin components being anti-symmetric. Then when dealing with multiple fermions, say two electrons, this means the spin component may either by a spin singlet (symmetric) or a spin triplet (anti-symmetric).

\noindent In order to form an anti-symmetric eigenstate for N non-interacting fermions within a common potential, we calculate the Slater determinant of an $n \times N$ matrix, where n is the number of eigenstates available for the N fermions. The Slater determinant looks something like:

\begin{equation}
    \renewcommand*{\arraystretch}{1.5}
    \setlength\arraycolsep{20pt}
    \Psi^{AS} (1, 2, ..., N) = \frac{1}{\sqrt{N!}} \begin{vmatrix} \phi_{E_a\sigma_a} (1) \hspace{2mm} \phi_{E_a\sigma_a} (2) \hspace{2mm} ... \hspace{2mm} \phi_{E_a\sigma_a} (N) \\ \phi_{E_b\sigma_b} (1) \hspace{2mm} \phi_{E_b\sigma_b} (2) \hspace{2mm} ... \hspace{2mm} \phi_{E_b\sigma_b} (N) \\ ... \\ \phi_{E_n\sigma_n} (1) \hspace{2mm} \phi_{E_n\sigma_n} (2) \hspace{2mm} ... \hspace{2mm} \phi_{E_n\sigma_n} (N)      \end{vmatrix}
    \label{SlaterDeterminant}
\end{equation}

\noindent where $\frac{1}{\sqrt{N!}}$ is a normalisation constant and $\phi_{E_n\sigma_n} (N)$, is the eigenstate with energy $E_n$ and spin $\sigma_n$ that the $N^{th}$ fermion occupies. It's important to note that if we interchange two columns, this is equivalent to applying the exchange operator, and the sign of the determinant changes. Also if we attempt to put two fermions into exactly the same quantum state (same energy level and same spin) then the determinant vanishes, as we would expect by the PEP.

\noindent For interest, the way you calculate the symmetric eigenstate for N non-interacting bosons within a common potential is found by calculating the Slater permanent from the same $n \times N$ matrix, as Eq~\ref{SlaterDeterminant}. The only difference between the permanent and determinant of a matrix is that you don't have to worry about the negative signs in the co-factors when calculating the permanent.

\section{Key Results}

In order for an eigenstate of an operator, which corresponds to an observable, to be physically acceptable it must also be an eigenstate of the particle exchange operator.

\noindent Fermions are particles with half integer spin, they are described by Fermi-Dirac statistics, Eq~\ref{FermiDiracDistributionFunction}. Their overall eigenstate must be anti-symmetric, this is apparent when operated on by the particle exchange operator. This eigenstate can be calculated using the Slater determinant, Eq~\ref{SlaterDeterminant}. Fermions must obey the PEP, which prevents two fermions from occupying the exact same quantum state.

\noindent Bosons are particles with integer spin, they are desribed by Bose-Einstein statistics, Eq~\ref{BoseEinsteinDistributionFunction}. Their overall eigenstate must symmetric, when operated on by the particle exchange operator.

\noindent Changing the order of particle labels should have no measurable difference on the observable values that the operator return, this is exchange degeneracy.

\section{Exam Question?}



\chapter{Multi electron atoms and the periodic table}
\label{chapt6}



\chapter{Time dependent PT}
\label{chapt7}

\section{Introduction}

Up to this point all the systems we have solved are time independent, i.e the Hamiltonian has no time dependence $\Hat{H} \neq \Hat{H}(t)$, such that the potential energy operator, $\Hat{V}$, is only a function of position and not time: $\Hat{V} = \Hat{V}(\underline{r}).$ This is known as quantum statics. This means that the Time Dependent Schrodinger Equation (TDSE):

\begin{equation}
    i\hbar \frac{\partial}{\partial t} \ket{\Psi(\underline{r}, t)} = \Hat{H}(\underline{r}) \ket{\Psi(\underline{r}, t)}
    \label{TDSE},
\end{equation}

\noindent can be solved using separation of variables, such that we can write $\Psi(\underline{r}, t) = \psi(\underline{r}) \phi(t)$, where $\psi(\underline{r})$ satisfies the Time Independent Schrodinger Equation (TISE):

\begin{equation}
    \Hat{H}(\underline{r}) \ket{\Psi(\underline{r})} = E \ket{\Psi(\underline{r})}
    \label{TISE}
\end{equation}

\noindent and $\phi(t) = e^{\frac{-iEt}{\hbar}}$, with $E$ the energy eigenvalue corresponding to the eigenstate $\ket{\Psi(\underline{r})}$. However in this section we consider some basic quantum dynamics, i.e $\Hat{V} = \Hat{V}(\underline{r}, t)$.

\section{Derivation}

\subsection{Non-Stationary States}

Recall that for a quantum static system, if the system occupies an eigenstate of the time independent Hamiltonian, $\Hat{H}_0$, then it is said to be in a stationary state of the system. This just means that no matter what time you measure the system it is always in the same eigenstate with the same eigenvalue, i.e initially it's in the state $\ket{\psi_m}$ and then at some later time $t$ in state $\ket{\psi_m} e^\frac{-iE_m t}{\hbar}$ which are equivalent once a measurement is applied.

\noindent However now consider what happens when the same quantum static system is not in a stationary state of the system, $\ket{\Psi(\underline{r}, t)}$. Suppose at $t = 0$, the state of the system may be written as a linear combination of the complete set of eigenstates that the system can occupy:

\begin{equation}
    \ket{\Psi(\underline{r}, t = 0)} = \sum_n = c_n \ket{\psi_n},
    \label{LinearCombinationStatesNonStationaryNoTimeDependence}
\end{equation}

\noindent such that:

\begin{equation}
    \ket{\Psi(\underline{r}, t)} = \sum_n = c_n \ket{\psi_n} e^\frac{-i E_n t}{\hbar}.
    \label{LinearCombinationStatesNonStationaryTimeDependence}
\end{equation}

\noindent This linear combination of stationary states means that we have a system, described by a Hamiltonian which is inherently time independent, that can undergo transitions (or quantum jumps) between different states assuming the system doesn't start in a stationary state.

\noindent This should become more clear after the following 2-level system example. Consider a system with Hamiltonian, $\hat{H}_0$, which is initially in the state:

\begin{equation}
    \ket{\Psi(\underline{r}, t = 0)} = c_m \ket{\psi_m} + c_n \ket{\psi_n},
    \label{TwoLevelSystemInitialState}
\end{equation}

\noindent where $\ket{\psi_m}$ and $\ket{\psi_n}$ are the orthonormal eigenstates of the system, with eigenvalues $E_m$ and $E_n$ respectively. This means that in the absence of any perturbation we have:

\begin{equation}
    \ket{\Psi(\underline{r}, t)} = c_m \ket{\psi_m} e^\frac{-i E_m t}{\hbar} + c_n \ket{\psi_n} e^\frac{-i E_n t}{\hbar},
    \label{TwoLevelSystemGeneralState}
\end{equation}

\noindent where we note that normalisation requires $|c_m|^2 + |c_n|^2 = 1$. Consider the expectation value of this system with respect to some arbitrary time independent operator $\Hat{A}$:

\begin{equation}
    \bra{\psi}\Hat{A}\ket{\psi} = \langle a \rangle _t = |c_m|^2 A_{mm} + |c_n|^2 A_{nn} + 2 \Re e[c_m^* c_n A_{mn} e^{i\omega t}],
    \label{TwoLevelSystemExpectationValue}
\end{equation}

\noindent where $\ket{\psi}$ is given by Eq~\ref{TwoLevelSystemGeneralState}, $\omega = \frac{E_m - E_n}{\hbar}$, $A_{mn} = \bra{\psi_m}\Hat{A}\ket{\psi_n}$ and we have used the hermiticity of $\Hat{A}$. Note the subscript $t$ on $\langle a \rangle$ denotes that the expectation value varies with time. From Eq~\ref{TwoLevelSystemExpectationValue} we can see that $\langle a \rangle _t$ oscillates with a period $T = \frac{2\pi}{\omega}$ between:

\begin{equation}
    |c_m|^2 A_{mm} + |c_n|^2 A_{nn} \pm 2 c_m^* c_n A_{mn},
\end{equation}

\noindent since $\Re e[e^{i \omega t}] = \cos(\omega t)$. This oscillation represents the transitions or quantum jumps between eigenstates.

\subsection{Time Dependent Perturbation Theory}

In general, to allow the quantum jumps between energy eigenstates we saw in the previous section we can allow $\Hat{H} = \Hat{H}(\underline{r}, t)$. This is normally done by introducing a time dependent potential, $\Hat{V}(t)$, which we can treat as a perturbation if $\Hat{V}(t) \ll \Hat{H}_0$. This is known as time dependent perturbation theory. This means that the Hamiltonian of the system will now take the form:

\begin{equation}
    \Hat{H} = \Hat{H}_0 + \Hat{V}(t)
    \label{TimeDependentHamiltonian}
\end{equation}

\noindent If we introduce a time dependent perturbation, $\Hat{V}(t)$, to our 2 level system, Eq~\ref{TwoLevelSystemGeneralState} becomes:

\begin{equation}
    \ket{\Psi(\underline{r}, t)} = c_m(t) \ket{\psi_m} e^\frac{-i E_m t}{\hbar} + c_n(t) \ket{\psi_n} e^\frac{-i E_n t}{\hbar},
    \label{TwoLevelPerturbedSystemGeneralState}
\end{equation}

\noindent where the coefficients are now functions of time. We wish to determine the form of these coefficients. We can generalise this argument easily to an $n$ level system with $n$ eigenstates. We can solve for each $c_n(t)$ by demanding that our linear superposition of states, analogous to Eq~\ref{TwoLevelPerturbedSystemGeneralState}: 

\begin{equation}
    \ket{\Psi(\underline{r}, t)} = \sum_n c_n(t) \ket{\psi_n} e^\frac{-i E_n t}{\hbar}
    \label{GeneralTimeDependentPerturbationWavefunction}
\end{equation}

\noindent obeys the TDSE, Eq~\ref{TDSE}, which results in:

\begin{equation}
    i \hbar \sum_n \Dot{c}_n(t) \ket{\psi_n} e^\frac{-i E_n t}{\hbar} = \sum_n \Hat{V}(t) \ket{\psi_n} e^\frac{-i E_n t}{\hbar} c_n(t).
\end{equation}

\noindent By pre-multiplying the above equation by $\bra{\psi_p}$, i.e take the inner product, we find:

\begin{equation}
    i \hbar \Dot{c}_p(t) = \sum_n V_{pn}(t) e^{i \omega_{pn} t} c_n(t)
    \label{TDSECondition}
\end{equation}

\noindent where $V_{pn} = \bra{\psi_p} \Hat{V} \ket{\psi_n}$ and $\omega_{pn} = \frac{E_p - E_n}{\hbar}$, which is known as the Bohr frequency. Note on the LHS of the equation we have used the orthonormality of the eigenstates to remove the summation. Up to this point we haven't actually applied any perturbative expansions, all of this is exact. However we now assume $\Hat{V}(t)$ is small and that the system begins in an eigenstate of the system, i.e $c_p(0) = \delta_{pn}$, so that to first order we find from Eq~\ref{TDSECondition}:

\begin{equation}
    i \hbar \Dot{c}^{(1)}_p(t) = V_{pn}(t) e^{i \omega_{pn} t},
    \label{FirstOrderTimeDependentPT}
\end{equation}

\noindent where the superscript on the coefficient represents to first order. We find this equation from substituting the zeroth order values onto the RHS of Eq~\ref{TDSECondition}. Now utilising the initial state condition above and Eq~\ref{FirstOrderTimeDependentPT} we find for $p = n$:

\begin{equation}
    \boxed{c^{1}_n(\tau) = 1 + \frac{1}{i \hbar} \int_0^\tau V_{nn}(t) dt},
    \label{CoefficientN}
\end{equation}

\noindent and:

\begin{equation}
    \boxed{c^{1}_p(\tau) = \frac{1}{i \hbar} \int_0^\tau V_{pn}(t)e^{i\omega_{pn}t} dt}.
    \label{CoefficientP}
\end{equation}

\noindent Here we interpret $|c^{1}_p(\tau)|^2$ as the probability to first order that after some time $\tau$ the system has performed a transition from state $\ket{\psi_n}$ to $\ket{\psi_p}$. Note that for first order PT to be valid we require:

\begin{equation}
    P^{(1)}_{n \rightarrow p}(\tau) = |c^{1}_p(\tau)|^2 \ll 1,
    \label{ConditionFirstOrderPT}
\end{equation}

\noindent i.e that the probability of such a transition occurring is much smaller compared to the probability of the system remaining in its initial state.

\noindent We now consider a specific example where the perturbation is sinusoidal in time:

\begin{equation}
    \Hat{V}(\underline{r}, t) = \Hat{V}(\underline{r}) \cos(\omega t)
    \label{SinusoidalPerturbation}.
\end{equation}

\noindent Using this form of the perturbation in Eq~\ref{CoefficientP} and considering only driving frequencies, $\omega$, that are close to the resonant Bohr frequency, $\omega_{pn}$ we find:

\begin{equation}
    c^{(1)}_p(\tau) = -i \frac{V_{pn}}{\hbar} \frac{\sin[(\omega_{pn} - \omega)\frac{\tau}{2}]}{\omega_{pn} - \omega} e^{i(\omega_{pn} - \omega) \frac{\tau}{2}},
    \label{SinusoidalPerturbationCoefficient}
\end{equation}

\noindent therefore the resulting probability is:

\begin{equation}
    P^{(1)}_{n \rightarrow p}(\tau) = \frac{{V_{pn}}^2}{\hbar^2} \frac{\sin^2[(\omega_{pn} - \omega)\frac{\tau}{2}]}{{(\omega_{pn} - \omega})^2}
    \label{SinusoidalPerturbationProbability}.
\end{equation}

\noindent Note how this probability oscillates with time. Further note, for future reference, that the approximation of considering only $\omega \simeq \omega_{pn}$ is known as the rotating wave approximation.

\noindent We now consider the case where the system isn't in a stationary state and our perturbation is time independent, $\Hat{V} \neq \Hat{V}(t)$. So transitions will occur, as before in Section 7.2.1, which means we can use time dependent PT. Then from Eq~\ref{CoefficientP} we find that:

\begin{equation}
    c^{(1)}_p(\tau) = \frac{V_{pn}}{\hbar \omega_{pn}} (1 - e^{i \omega_{pn} \tau}),
\end{equation}

\noindent and the resulting probability:

\begin{equation}
    P^{(1)}_{n \rightarrow p}(\tau) = \frac{4 |V_{pn}|^2}{\hbar^2} \frac{\sin^2(\frac{\omega_{pn} \tau}{2})}{\omega_{pn}^2}.
\end{equation}

\noindent Time dependent PT can also be used for when the final states are not discrete and well separated - i.e they form a continuum. The number of states within some interval is given by $\rho(E_p) dE_p$, then we find:

\begin{equation}
    P^{(1)}_{n \rightarrow G}(\tau) = \tau \frac{2\pi}{\hbar}[|V_{pn}|^2 \rho(E_p)]_{E_p = E_n}.
\end{equation}

\noindent It's important to note that this is proportional to $\tau$, so the transition probability per unit time is given by:

\begin{equation}
    \boxed{R^{(1)}_{n \rightarrow G}(\tau) = \frac{2\pi}{\hbar}[|V_{pn}|^2 \rho(E_p)]_{E_p = E_n}}.
    \label{FermiGoldenRuleContinuum}
\end{equation}

\noindent This is known as Fermi's Golden Rule. The equivalent Fermi's Golden Rule for transitions between 2 discrete states is:

\begin{equation}
    \boxed{R^{(1)}_{n \rightarrow G}(\tau) = \frac{2\pi}{\hbar}|V_{pn}|^2 \delta(E_p - E_n)}.
\end{equation}

\subsection{Emission and Absorption of EM radiation}

We are now going to use time dependent PT for the emission and absorption of EM radiation by an atom. The atom primarily interacts with the electric component of the EM field:

\begin{equation}
    \underline{E}(\underline{r}, t) = E_0 \underline{\Hat{n}}\cos(\underline{k} \cdot \underline{r} - \omega t),
\end{equation}

\noindent where assuming the wavelength of EM radiation is much larger compared to the size of the atom we can neglect the spatial variation of the field, this is known as the dipole approximation, and results in the electric field taking the form:

\begin{equation}
    \underline{E}(\underline{r}, t) = E_0 \underline{\Hat{n}}\cos(\omega t).
    \label{DipoleApproximationElectricField}
\end{equation}

\noindent This means that our perturbation takes the form:

\begin{equation}
    \Hat{V}(t) = \underline{\Hat{D}} \cdot \underline{E}(t) = \frac{1}{2} \Hat{D}_n E_0 (e^{i\omega t} + e^{-i\omega t})
    \label{PerturbationEM},
\end{equation}

\noindent where $\underline{\Hat{D}}$ is the electric dipole operator of the atom:

\begin{equation}
    \underline{\Hat{D}} = -q \sum_{j = 1}^Z \underline{\Hat{r}}_j
    \label{ElectricDipoleOperator}
\end{equation}

\noindent This is effectively a sum over all the positions of the electrons that surround an atom with atomic number, $Z$. Then following the same procedure as before we find:

\begin{equation}
    c^{(1)}_p(\tau) = \frac{E_0}{2i\hbar}\bra{\psi_p} \Hat{D}_n \ket{\psi_n}\left[\frac{e^{i(\omega_{pn} - \omega)\tau} - 1}{i(\omega_{pn} - \omega)} + \frac{e^{i(\omega_{pn} + \omega)\tau} - 1}{i(\omega_{pn} + \omega)}\right].
    \label{CoefficientEM}
\end{equation}

\noindent We now have 2 cases to consider. The first is $\omega = \omega_{pn}$ which corresponds to stimulated absorption. So the probability of an incoming photon causing an electron to move from a lower energy state to a higher energy state is:

\begin{equation}
    P^{(1)}_{n\rightarrow p}(\tau) = \frac{E_0^2}{\hbar^2} |\bra{\psi_p} \Hat{D}_n \ket{\psi_n}|^2 \frac{\sin^2[(\omega_{pn} - \omega)\frac{\tau}{2}]}{(\omega_{pn} - \omega)^2}.
    \label{StimulatedAbsorption}
\end{equation}

\noindent The second case is that of $\omega = -\omega_{pn}$ which corresponds to stimulated emission. So the probability of an incoming photon causing an electron to move from a higher energy state to a lower energy state by emitting another photon is given by:

\begin{equation}
    P^{(1)}_{p\rightarrow n}(\tau) = \frac{E_0^2}{\hbar^2} |\bra{\psi_n} \Hat{D}_n \ket{\psi_p}|^2 \frac{\sin^2[(\omega_{pn} - \omega)\frac{\tau}{2}]}{(\omega_{pn} - \omega)^2}.
    \label{StimulatedEmission}
\end{equation}

\noindent There is a third mechanism for radiation interacting with matter, known as spontaneous emission. Here even if an atom is in an excited state and there are no incident photons, the electron can still move to a lower energy state. This is because due to QED there are still fields present that the atom can interact with which can cause this photon emission even in a vacuum state.

\section{Key Results}

\chapter{Lasers}
\label{chapt8}

\section{Introduction}

\section{Derivation}

\subsection{Einstein Coefficients}

\subsection{Time Dependent PT}

\subsection{Components and Operation of LASER}

\section{Key Results}

\section{Exam Question?}

\chapter{Two level systems}
\label{chapt9}

\section{Introduction}

\section{Ammonia Molecule}

\section{General Undriven 2-level system}

\section{General Driven 2-level system}

\section{Bloch Sphere Visualisation}

\section{Magnetic Resonace}

\section{Key Results}

\include{chap10}
\include{chap11}
\include{chap12}


%----------------------------------------------------------------------------------------
%	THESIS CONTENT - APPENDICES
%----------------------------------------------------------------------------------------

\appendix % Cue to tell LaTeX that the following "chapters" are Appendices
\renewcommand{\thetable}{A\arabic{chapter}.\arabic{table}} % adds an A to table names in appendix (Table A1.1, A1.2...)
\renewcommand{\thefigure}{A\arabic{chapter}.\arabic{figure}} % same for figures
\renewcommand{\thesection}{A\arabic{section}}

% Include the appendices of the thesis as separate files from the Appendices folder
% \input{Appendix/Supp_Chap3.tex}


% \backmatter
% %% A list of publications can be created using this approach
% \include{ownpubs}

\end{document}
